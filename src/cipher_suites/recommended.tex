%%\subsection{Recommended cipher suites}

In principle system administrators who want to improve their communication security
have to make a difficult decision between effectively locking out some users and 
keeping high cipher suite security while supporting as many users as possible.
The website \url{https://www.ssllabs.com/} gives administrators and security engineers
a tool to test their setup and compare compatibility with clients. The authors made 
use of ssllabs.com to arrive at a set of cipher suites which we will recommend 
throughout this document.\\

\textbf{Caution: these settings can only represent a subjective
choice of the authors at the time of writing. It might be a wise choice to
select your own and review cipher suites based on the instructions in section
\ref{section:ChoosingYourOwnCipherSuites}}.


\subsubsection{Configuration A: Strong ciphers, fewer clients}

At the time of writing we recommend the following set of strong cipher
suites which may be useful in an environment where one does not depend on many,
different clients and where compatibility is not a big issue.  An example
of such an environment might be machine-to-machine communication or corporate
deployments where software that is to be used can be defined freely.


We arrived at this set of cipher suites by selecting:

\begin{itemize}
\item TLS 1.2
\item Perfect forward secrecy / ephemeral Diffie Hellman
\item strong MACs (SHA-2) or
\item GCM as Authenticated Encryption scheme
\end{itemize}

This results in the OpenSSL string:

\begin{lstlisting}[breaklines]
'EECDH+aRSA+AES256:EDH+aRSA+AES256:!SSLv3'
\end{lstlisting}

%$\implies$ resolves to 
%
%\begin{verbatim}
%openssl ciphers -V $string
%\end{verbatim}



%\todo{make a column for cipher chaining mode} --> not really important, is it?
\begin{center}

\begin{tabular}{lllllll}
\toprule
\textbf{ID}   & \textbf{OpenSSL Name}       & \textbf{Version} & \textbf{KeyEx} & \textbf{Auth} & \textbf{Cipher} & \textbf{MAC}\\\cmidrule(lr){1-7}
\verb|0xC030| & ECDHE-RSA-AES256-GCM-SHA384 & TLSv1.2          & ECDH           &  RSA          & AESGCM(256)     & AEAD         \\
\verb|0xC028| & ECDHE-RSA-AES256-SHA384     & TLSv1.2          & ECDH           &  RSA          & AES(256) (CBC)  & SHA384       \\
\verb|0x009F| & DHE-RSA-AES256-GCM-SHA384   & TLSv1.2          & DH             &  RSA          & AESGCM(256)     & AEAD         \\
\verb|0x006B| & DHE-RSA-AES256-SHA256       & TLSv1.2          & DH             &  RSA          & AES(256) (CBC)  & SHA256       \\
\bottomrule
\end{tabular}
\end{center}


\textbf{Compatibility}

Only clients which support TLS 1.2 are covered by these cipher suites (Chrome 30,
Win 7 and Win 8.1 crypto stack, Opera 17, OpenSSL $\ge$ 1.0.1e, Safari 6 / iOS
6.0.1, Safari 7 / OS X 10.9).



\subsubsection{Configuration B: Weaker ciphers, more compatability}

In this section we propose a slightly weaker set of cipher suites.  For
example, there are some known weaknesses for the SHA-1 hash function that is
included in this set.  The advantage of this set of cipher suites is not only
the compatibility with a broad range of clients, but also less computational
workload on the provisioning hardware.
\\

\textbf{All further examples in this publication use Configuration B}.\\

We arrived at this set of cipher suites by selecting:

\begin{itemize}
\item TLS 1.2, TLS 1.1, TLS 1.0
\item allow SHA-1

\todo{AK: Note that SHA1 is considered broken but if we are in DHE, we might get around it as long as you can not calculate a SHA1 collision ``live'' on the wire}

\end{itemize}

This results in the OpenSSL string:

\begin{lstlisting}[breaklines]
'EDH+CAMELLIA:EDH+aRSA:EECDH+aRSA+AESGCM:EECDH+aRSA+SHA384:EECDH+aRSA+SHA256:EECDH:+CAMELLIA256:+AES256:+CAMELLIA128:+AES128:+SSLv3:!aNULL:!eNULL:!LOW:!3DES:!MD5:!EXP:!PSK:!SRP:!DSS:!RC4:!SEED:!ECDSA:CAMELLIA256-SHA:AES256-SHA:CAMELLIA128-SHA:AES128-SHA'
\end{lstlisting}

\todo{make a column for cipher chaining mode}
\begin{center}
\begin{tabular}{lllllll}
\toprule
\textbf{ID}   & \textbf{OpenSSL Name}       & \textbf{Version} & \textbf{KeyEx} & \textbf{Auth} & \textbf{Cipher} & \textbf{MAC}\\\cmidrule(lr){1-7}
\verb|0x009F| & DHE-RSA-AES256-GCM-SHA384   & TLSv1.2          & DH             & RSA           & AESGCM(256)     & AEAD         \\
\verb|0x006B| & DHE-RSA-AES256-SHA256       & TLSv1.2          & DH             & RSA           & AES(256)        & SHA256       \\
\verb|0xC030| & ECDHE-RSA-AES256-GCM-SHA384 & TLSv1.2          & ECDH           & RSA           & AESGCM(256)     & AEAD         \\
\verb|0xC028| & ECDHE-RSA-AES256-SHA384     & TLSv1.2          & ECDH           & RSA           & AES(256)        & SHA384       \\
\verb|0x009E| & DHE-RSA-AES128-GCM-SHA256   & TLSv1.2          & DH             & RSA           & AESGCM(128)     & AEAD         \\
\verb|0x0067| & DHE-RSA-AES128-SHA256       & TLSv1.2          & DH             & RSA           & AES(128)        & SHA256       \\
\verb|0xC02F| & ECDHE-RSA-AES128-GCM-SHA256 & TLSv1.2          & ECDH           & RSA           & AESGCM(128)     & AEAD         \\
\verb|0xC027| & ECDHE-RSA-AES128-SHA256     & TLSv1.2          & ECDH           & RSA           & AES(128)        & SHA256       \\
\verb|0x0088| & DHE-RSA-CAMELLIA256-SHA     & SSLv3            & DH             & RSA           & Camellia(256)   & SHA1         \\
\verb|0x0039| & DHE-RSA-AES256-SHA          & SSLv3            & DH             & RSA           & AES(256)        & SHA1         \\
\verb|0xC014| & ECDHE-RSA-AES256-SHA        & SSLv3            & ECDH           & RSA           & AES(256)        & SHA1         \\
\verb|0x0045| & DHE-RSA-CAMELLIA128-SHA     & SSLv3            & DH             & RSA           & Camellia(128)   & SHA1         \\
\verb|0x0033| & DHE-RSA-AES128-SHA          & SSLv3            & DH             & RSA           & AES(128)        & SHA1         \\
\verb|0xC013| & ECDHE-RSA-AES128-SHA        & SSLv3            & ECDH           & RSA           & AES(128)        & SHA1         \\
\verb|0x0084| & CAMELLIA256-SHA             & SSLv3            & RSA            & RSA           & Camellia(256)   & SHA1         \\
\verb|0x0035| & AES256-SHA                  & SSLv3            & RSA            & RSA           & AES(256)        & SHA1         \\
\verb|0x0041| & CAMELLIA128-SHA             & SSLv3            & RSA            & RSA           & Camellia(128)   & SHA1         \\
\verb|0x002F| & AES128-SHA                  & SSLv3            & RSA            & RSA           & AES(128)        & SHA1         \\
\bottomrule
\end{tabular}
\end{center}

\textbf{Compatibility}

Note that these cipher suites will not work with Windows XP's crypto stack (e.g. IE, Outlook), 
%%Java 6, Java 7 and Android 2.3. Java 7 could be made compatible by installing the "Java 
%%Cryptography Extension (JCE) Unlimited Strength Jurisdiction Policy Files"
%%(JCE) \footnote{\url{http://www.oracle.com/technetwork/java/javase/downloads/jce-7-download-432124.html}}.
We could not verify yet if installing JCE also fixes the Java 7
DH-parameter length limitation (1024 bit). 
\todo{do that!}

\textbf{Explanation}

For a detailed explanation of the cipher suites chosen, please see
\ref{section:ChoosingYourOwnCipherSuites}. In short, finding the perfect cipher
string is impossible and must be a tradeoff between compatibility and security. 
On the one hand there are mandatory and optional ciphers defined in a few RFCs, 
on the other hand there are clients and servers only implementing subsets of the 
specification.

Straight forward, the authors wanted strong ciphers, forward secrecy
\footnote{\url{http://nmav.gnutls.org/2011/12/price-to-pay-for-perfect-forward.html}}
and the best client compatibility possible while still ensuring a cipher string that can be
used on legacy installations (e.g. OpenSSL 0.9.8). 

Our recommended cipher strings are meant to be used via copy and paste and need to work
"out of the box".

\begin{itemize}
\item TLSv1.2 is preferred over TLSv1.0/SSLv3 (while still providing a useable cipher
      string for SSLv3).
\item AES256 and CAMELLIA256 count as very strong ciphers at the moment; preferrably in
      GCM mode.\\
	  \todo{add a reference here please}
\item AES128 and CAMELLIA128 count as strong enough ciphers at the moment
\item DHE or ECDHE for forward secrecy
\item RSA as this will fit most of todays setups
\item AES256-SHA as a last resort (with this cipher at the end, even systems with
      very old versions of openssl like 0.9.8 will just work. Just forward secrecy
      will not be used. On systems that do not support elliptic curves, that cipher
      offers support for the Microsoft crypto libraries that only support ECDHE.
\end{itemize}

\todo{Adi: review "justification" when next section is written}
