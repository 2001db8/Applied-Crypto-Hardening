%%\subsection{Webservers}

\subsection{Apache}

\subsubsection{description}
\subsubsection[Tested with Version:]

\subsubsection[Settings:] \mbox{}

%-All +TLSv1.1 +TLSv1.2
\begin{lstlisting}[breaklines]
  SSLProtocol All -SSLv2 -SSLv3 
  SSLHonorCipherOrder On
  SSLCompression off
  # Add six earth month HSTS header for all users...
  Header add Strict-Transport-Security "max-age=15768000"
  # If you want to protect all subdomains, use the following header
  # ALL subdomains HAVE TO support https if you use this!
  # Strict-Transport-Security: max-age=15768000 ; includeSubDomains

  SSLCipherSuite '@@@CIPHERSTRINGB@@@'

\end{lstlisting}

Note again, that any cipher suite starting with ECDHE can be omitted, if in doubt.
%% XXX NOTE TO SELF: remove from future automatically generated lists!

\subsubsection[Additional settings:]

You should redirect everything to httpS:// if possible. In Apache you can do this with the following setting inside of a VirtualHost environment:

\begin{lstlisting}[breaklines]
  <VirtualHost *:80>
   #...
   RewriteEngine On
        RewriteRule ^.*$ https://%{SERVER_NAME}%{REQUEST_URI} [L,R=permanent]
   #...
  </VirtualHost>
\end{lstlisting}

\subsubsection[Justification for special settings (if needed):]

\subsubsection[References:]

\subsubsection[How to test:]

See ssllabs in section \ref{section:Tools}

%\end{description}


\subsection{lighttpd}



\begin{description}
\item[Tested with Version:]

\todo{version?}

\item[Settings:] \mbox{}


%% Complete ssl.cipher-list with same algo than Apache
\todo{FIXME: this string seems to be wrongly formatted??}

\begin{lstlisting}[breaklines]
  $SERVER["socket"] == "0.0.0.0:443" {
    ssl.engine  = "enable"
    ssl.use-sslv2 = "disable"
    ssl.use-sslv3 = "disable"
    #ssl.use-compression obsolete >= 1.4.3.1
    ssl.pemfile = "/etc/lighttpd/server.pem"
    ssl.cipher-list = 'EECDH+aRSA+AESGCM:EECDH+aRSA+SHA384:EECDH+aRSA+SHA256:EDH+CAMELLIA256:EECDH:EDH+aRSA:+SSLv3:!aNULL:!eNULL:!LOW:!3DES:!MD5:!EXP:!PSK:!SRP:!DSS:!RC4:!SEED:!AES128:!CAMELLIA128:!ECDSA:AES256-SHA'
    ssl.honor-cipher-order = "enable"
    setenv.add-response-header  = ( "Strict-Transport-Security" => "max-age=31536000")
  }
\end{lstlisting}


\item[Additional settings:]

As for any other webserver, you should automatically redirect http traffic toward httpS://

\begin{lstlisting}[breaklines]
  $HTTP["scheme"] == "http" {
    # capture vhost name with regex conditiona -> %0 in redirect pattern
    # must be the most inner block to the redirect rule
    $HTTP["host"] =~ ".*" {
        url.redirect = (".*" => "https://%0$0")
    }
  }
\end{lstlisting}


\item[References:] 
\todo{add references}.
lighttpd httpS:// redirection: \url{http://redmine.lighttpd.net/projects/1/wiki/HowToRedirectHttpToHttps}

% add any further references or best practice documents here

\item[How to test:] See ssllabs in section \ref{section:Tools}

% describe here or point the admin to tools (can be a simple footnote or \ref{} to  the tools section) which help the admin to test his settings.
\end{description}


\subsection{nginx}

\begin{description}
\item[Tested with Version:] 1.4.4 with OpenSSL 1.0.1e on OS X Server 10.8.5

1.2.1-2.2+wheezy2 with OpenSSL 1.0.1e on Debian Wheezy
1.4.4 with OpenSSL 1.0.1e on Debian Wheezy
1.4.4 with OpenSSL 1.0.1e on Debian Wheezy
1.4.4 with OpenSSL 1.0.1e on Debian Wheezy
1.4.4 with OpenSSL 1.0.1e on Debian Wheezy

\todo{version?}

\item[Settings:] \mbox{}

\begin{lstlisting}[breaklines]
  ssl_prefer_server_ciphers on;
  ssl_protocols TLSv1 TLSv1.1 TLSv1.2;
  ssl_ciphers 'EECDH+aRSA+AESGCM:EECDH+aRSA+SHA384:EECDH+aRSA+SHA256:EDH+CAMELLIA256:EECDH:EDH+aRSA:+SSLv3:!aNULL:!eNULL:!LOW:!3DES:!MD5:!EXP:!PSK:!SRP:!DSS:!RC4:!SEED:!AES128:!CAMELLIA128:!ECDSA:AES256-SHA';
  add_header Strict-Transport-Security max-age=2592000;
\end{lstlisting}

%% XXX FIXME: do we need to specify dhparams? Parameter: ssl_dhparam = file. See: http://wiki.nginx.org/HttpSslModule#ssl_protocols
%% NO, use IETF/IKE

If you absolutely want to specify your own DH parameters, you can specify them via

\begin{lstlisting}[breaklines]
  ssl_dhparam file;
\end{lstlisting}

However, we advise you to read section \ref{section:DH} and stay with the standard IKE/IETF parameters (as long as they are $ > 1024 $ bits).


\item[Additional settings:]

If you decide to trust NIST's ECC curve recommendation, you can add the following line to nginx's configuration file to select special curves:

\begin{lstlisting}[breaklines]
  ssl_ecdh_curve          secp384r1;
\end{lstlisting}

You should redirect everything to httpS:// if possible. In Nginx you can do this with the following setting:

\begin{lstlisting}[breaklines]
  rewrite     ^(.*)   https://$host$1 permanent;
\end{lstlisting}


\item[References:] \todo{add references}

\item[How to test:] See ssllabs in section \ref{section:Tools}

\end{description}





\subsection{MS IIS}
\label{sec:ms-iis}


\todo{Daniel: add screenshots and registry keys}

\begin{description}

\item[Tested with Version:] \todo{Daniel: add tested version}

\item[Settings:] \mbox{}


When trying to avoid RC4 and CBC (BEAST-Attack) and requiring perfect
forward secrecy, Microsoft Internet Information Server (IIS) supports
ECDSA, but does not support RSA for key exchange (consider ECC suite
B doubts\footnote{\url{http://safecurves.cr.yp.to/rigid.html}}).

Since \verb|ECDHE_RSA_*| is not supported, a SSL certificate based on
elliptic curves needs to be used.

The configuration of cipher suites MS IIS will use, can be configured in one
of the following ways:
\begin{enumerate}
\item Group Policy \footnote{\url{http://msdn.microsoft.com/en-us/library/windows/desktop/bb870930(v=vs.85).aspx}}
\item Registry
\item IIS Crypto~\footnote{\url{https://www.nartac.com/Products/IISCrypto/}}
\end{enumerate}


Table~\ref{tab:MS_IIS_Client_Support} shows the process of turning on
one algorithm after another and the effect on the supported clients
tested using https://www.ssllabs.com.

\verb|SSL 3.0|, \verb|SSL 2.0| and \verb|MD5| are turned off.
\verb|TLS 1.0| and \verb|TLS 2.0| are turned on.

\begin{table}[h]
  \centering
  \small
  \begin{tabular}{ll}
    \toprule
    Cipher Suite & Client \\
    \midrule
    \verb|TLS_ECDHE_ECDSA_WITH_AES_128_GCM_SHA256| & only IE 10,11, OpenSSL 1.0.1e \\
    \verb|TLS_ECDHE_ECDSA_WITH_AES_128_CBC_SHA256| & Chrome 30, Opera 17, Safari 6+ \\
    \verb|TLS_ECDHE_ECDSA_WITH_AES_128_CBC_SHA| & FF 10-24, IE 8+, Safari 5, Java 7\\
    \bottomrule 
 \end{tabular}
  \caption{Client support}
  \label{tab:MS_IIS_Client_Support}
\end{table}

Table~\ref{tab:MS_IIS_Client_Support} shows the algoriths from
strongest to weakest and why they need to be added in this order. For
example insisting on SHA-2 algorithms (only first two lines) would
eliminate all versions of Firefox, so the last line is needed to
support this browser, but should be placed at the bottom, so capable
browsers will choose the stronger SHA-2 algorithms.

\verb|TLS_RSA_WITH_RC4_128_SHA| or equivalent should also be added if
MS Terminal Server Connection is used (make sure to use this only in a
trusted environment). This suite will not be used for SSL, since we do
not use a RSA Key.


% \verb|TLS_ECDHE_ECDSA_WITH_AES_128_GCM_SHA256| ... only supported by: IE 10,11, OpenSSL 1.0.1e
% \verb|TLS_ECDHE_ECDSA_WITH_AES_128_CBC_SHA256| ... Chrome 30, Opera 17, Safari 6+
% \verb|TLS_ECDHE_ECDSA_WITH_AES_128_CBC_SHA| ... Firefox 10-24, IE 8+, Safari 5, Java 7


Clients not supported:
\begin{enumerate}
\item Java 6
\item WinXP
\item Bing
\end{enumerate}

\item[Additional settings:]

%Here you can add additional settings

\item[Justification for special settings (if needed):]

% in case you have the need for further justifications why you chose this and that setting or if the settings do not fit into the standard Variant A or Variant B schema, please document this here

\item[References:]

\todo{add references}

% add any further references or best practice documents here

\item[How to test:] See ssllabs in section \ref{section:Tools}


\end{description}
