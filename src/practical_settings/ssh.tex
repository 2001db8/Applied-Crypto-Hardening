\subsubsection{OpenSSH}


\begin{description}
\item[Tested with Version:] OpenSSH 6.1

\item[Settings:] \mbox{}


\paragraph*{sshd_config}
\begin{lstlisting}[breaklines]
	# ...

	Protocol 2
	PermitEmptyPasswords no
	PermitRootLogin no
	StrictModes yes
	HostKey /etc/ssh/ssh_host_rsa_key
	ServerKeyBits 4096
	Ciphers aes256-gcm@openssh.com aes128-gcm@openssh.com aes256-ctr aes128-ctr
	MACs umac-128-etm@openssh.com,hmac-sha2-512,hmac-sha2-256,hmac-ripemd160
	KexAlgorithms curve25519-sha256@libssh.org,diffie-hellman-group-exchange-sha256,diffie-hellman-group14-sha1,diffie-hellman-group-exchange-sha1
\end{lstlisting}

% XXX: curve25519-sha256@libssh.org only available upstream(!)

Note: Older linux systems won't support SHA2. PuTTY (Windows) does not support
RIPE-MD160. Curve25519, AES-GCM and UMAC are only available upstream (OpenSSH
6.1). DSA host keys have been removed on purpose, the DSS standard does not
support for DSA keys stronger than 1024bit
\footnote{\url{https://bugzilla.mindrot.org/show_bug.cgi?id=1647}} which is far
below current standards (see section \ref{section:keylengths}). Legacy systems
can use this configuration and simply omit unsupported ciphers, key exchange
algorithms and MACs.  

\item[Additional settings:] \mbox{}

Note that the setting \texttt{ServerKeyBits 4096}  has no effect until you re-generate new ssh host keys. There might be issues if you have users which rely on the fingerprint of the old ssh host key being stored in their clients' \texttt{.ssh/known\_hosts} file.

\item[References:] The openssh sshd\_config  man page is the best reference: \url{http://www.openssh.org/cgi-bin/man.cgi?query=sshd_config}


\item[How to test:]
% describe here or point the admin to tools (can be a simple footnote or \ref{} to  the tools section) which help the admin to test his settings.

Connect with a client to an ssh server like this: \\
\begin{lstlisting}[breaklines]
$ ssh -vvv myserver.com
\end{lstlisting}
and observe the key exchange in the verbose output.

\end{description}
