%%---------------------------------------------------------------------- 
\subsection{OpenSSH}
\subsubsection{Tested with Version} OpenSSH 6.1
\subsubsection{Settings}
\paragraph*{sshd_config}
\begin{lstlisting}[breaklines]
	# ...

	Protocol 2
	PermitEmptyPasswords no
	PermitRootLogin no
	StrictModes yes
	HostKey /etc/ssh/ssh_host_rsa_key
	ServerKeyBits 4096
	Ciphers aes256-gcm@openssh.com aes128-gcm@openssh.com aes256-ctr aes128-ctr
	MACs umac-128-etm@openssh.com,hmac-sha2-512,hmac-sha2-256,hmac-ripemd160
	KexAlgorithms curve25519-sha256@libssh.org,diffie-hellman-group-exchange-sha256,diffie-hellman-group14-sha1,diffie-hellman-group-exchange-sha1
\end{lstlisting}

% XXX: curve25519-sha256@libssh.org only available upstream(!)
Note: Older linux systems won't support SHA2. PuTTY (Windows) does not support
RIPE-MD160. Curve25519, AES-GCM and UMAC are only available upstream (OpenSSH
6.1). DSA host keys have been removed on purpose, the DSS standard does not
support for DSA keys stronger than 1024bit
\footnote{\url{https://bugzilla.mindrot.org/show_bug.cgi?id=1647}} which is far
below current standards (see section \ref{section:keylengths}). Legacy systems
can use this configuration and simply omit unsupported ciphers, key exchange
algorithms and MACs.  
\subsubsection{Additional settings}
Note that the setting \texttt{ServerKeyBits 4096}  has no effect until you re-generate new ssh host keys. There might be issues if you have users which rely on the fingerprint of the old ssh host key being stored in their clients' \texttt{.ssh/known\_hosts} file.
%\subsubsection{Justification for special settings (if needed)}
\subsubsection{References}
The openssh sshd\_config  man page is the best reference: \url{http://www.openssh.org/cgi-bin/man.cgi?query=sshd_config}
\subsubsection{How to test}
Connect a client with verbose logging enabled to the SSH server \\
\begin{lstlisting}[breaklines]
$ ssh -vvv myserver.com
\end{lstlisting}and observe the key exchange in the output.


%%---------------------------------------------------------------------- 
\subsection{Cisco ASA}
\subsubsection{Tested with Version} 9.1(3)
\subsubsection{Settings}
\begin{lstlisting}[breaklines]
crypto key generate rsa modulus 2048
ssh version 2
ssh key-exchange group dh-group14-sha1
line vty 0 4
 transport input ssh
\end{lstlisting}
Note: When the ASA is configured for SSH, by default both SSH versions 1 and 2 are allowed. In addition to that, only a group1 DH-key-exchange is used. This should be changed to allow only SSH version 2 and to use a key-exchnage with group14. The generated RSA key should be 2048 bit (the actual supported maximum). A non-cryptographic best practice is to reconfigure the lines to only allow SSH-logins.
\subsubsection{References}
\url{http://www.cisco.com/en/US/docs/security/asa/asa91/configuration/general/admin\_management.html }
\subsubsection{How to test}
Connect a client with verbose logging enabled to the SSH server \\
\begin{lstlisting}[breaklines]
$ ssh -vvv myserver.com
\end{lstlisting}and observe the key exchange in the output.


%---------------------------------------------------------------------- 
\subsection{Cisco IOS}
\subsubsection{Tested with Version} 15.0, 15.1, 15.2
\subsubsection{Settings}
\begin{lstlisting}[breaklines]
crypto key generate rsa modulus 2048 label SSH-KEYS
ip ssh rsa keypair-name SSH-KEYS
ip ssh version 2
ip ssh dh min size 2048
\end{lstlisting}
Note: Same as with the ASA, also on IOS by default both SSH versions 1 and 2 are allowed and the DH-key-exchange only use a DH-group of 768 Bit.
In IOS, a dedicated Key-pair can be bound to SSH to reduce the usage of individual keys-pairs.
\subsubsection{References}
\url{http://www.cisco.com/en/US/docs/ios/sec\_user\_services/configuration/guide/sec\_secure\_shell\_v2.html }
% add any further references or best practice documents here
\subsubsection{How to test}
Connect a client with verbose logging enabled to the SSH server \\
\begin{lstlisting}[breaklines]
$ ssh -vvv myserver.com
\end{lstlisting}and observe the key exchange in the output.
