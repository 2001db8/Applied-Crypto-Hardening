%%\subsection{Database Systems}
% This list is based on : http://en.wikipedia.org/wiki/Relational_database_management_system#Market_share

\subsubsection{Oracle}
\item[Tested with Version:] not tested

\item[References:] (German)
{\small \url{www.telekom.com/static/-/155996/7/technische-sicherheitsanforderungen-si}}

Please read the following pages about SSL and ciphersuites:\\
p. 129 -Req 396 and Req 397 \\


\subsubsection{SQL Server}
\todo{write this}




\subsubsection{MySQL}

\begin{description}
\item[Tested with Version:] Debian 7.0 and MySQL 5.5

\item[Settings:] \mbox{}

\paragraph*{my.cnf}\mbox{}\\

\begin{lstlisting}[breaklines]
[mysqld]
ssl
ssl-ca=/etc/mysql/ssl/ca-cert.pem
ssl-cert=/etc/mysql/ssl/client-cert.pem
ssl-key=/etc/mysql/ssl/client-key.pem
ssl-cipher=EECDH+aRSA+AESGCM:EECDH+aRSA+SHA384:EECDH+aRSA+SHA256:EDH+CAMELLIA256:EECDH:EDH+aRSA:+SSLv3:!aNULL:!eNULL:!LOW:!3DES:!MD5:!EXP:!PSK:!SRP:!DSS:!RC4:!SEED:!AES128:!CAMELLIA128:!ECDSA:AES256-SHA
\end{lstlisting}

\item[Additional settings:]


\item[Justification for special settings (if needed):]

% in case you have the need for further justifications why you chose this and that setting or if the settings do not fit into the standard Variant A or Variant B schema, please document this here

\item[References:]

\todo{add references}

% add any further references or best practice documents here

\item[How to test:]

After restarting the server run the following query to see if the ssl settings are correct:
\begin{lstlisting}[breaklines]
show variables like '%ssl%';
\end{lstlisting}


\end{description}


\subsubsection{DB2}
\begin{description}
\item[Tested with Version:] not tested

\item[References:]
{\small \url{http://pic.dhe.ibm.com/infocenter/db2luw/v9r7/index.jsp?topic=%2Fcom.ibm.db2.luw.admin.sec.doc%2Fdoc%2Fc0053544.html}}


\paragraph*{ssl_cipherspecs}\mbox{}\\
In the link above the whole SSL-Configuration is in-depth described. The following command shows only the recommended ciphersuites.
\begin{lstlisting}[breaklines]
% it's out of scope to describe the whole SSL procedure
% # fully qualified path of the key database file
%db2 update dbm cfg using SSL_SVR_KEYDB /home/dba/sqllib/security/keystore/key.kdb
%
%# fully qualified path of the stash file
%db2 update dbm cfg using SSL_SVR_STASH /home/dba/sqllib/security/keystore/mydbserver.sth
%
%# label of the digital certificate of the server
%db2 update dbm cfg using SSL_SVR_LABEL myselfsigned
%
# recommended and supported ciphersuites 

db2 update dbm cfg using SSL_CIPHERSPECS 
TLS_RSA_WITH_AES_256_CBC_SHA256,
TLS_RSA_WITH_AES_128_GCM_SHA256,
TLS_RSA_WITH_AES_128_CBC_SHA256,
TLS_ECDHE_RSA_WITH_AES_128_GCM_SHA256,
TLS_ECDHE_ECDSA_WITH_AES_128_GCM_SHA256,
TLS_ECDHE_RSA_WITH_AES_128_CBC_SHA256,
TLS_ECDHE_ECDSA_WITH_AES_128_CBC_SHA256,
TLS_RSA_WITH_AES_256_GCM_SHA384,
TLS_ECDHE_RSA_WITH_AES_256_GCM_SHA384,
TLS_ECDHE_ECDSA_WITH_AES_256_GCM_SHA384,
TLS_ECDHE_RSA_WITH_AES_256_CBC_SHA384,
TLS_ECDHE_ECDSA_WITH_AES_256_CBC_SHA384,
TLS_ECDHE_RSA_WITH_AES_256_CBC_SHA,
TLS_ECDHE_ECDSA_WITH_AES_256_CBC_SHA,
TLS_RSA_WITH_AES_256_CBC_SHA,
TLS_RSA_WITH_AES_128_CBC_SHA,
TLS_ECDHE_RSA_WITH_AES_128_CBC_SHA,
TLS_ECDHE_ECDSA_WITH_AES_128_CBC_SHA

\end{lstlisting}


\subsubsection{Postgresql}

\begin{description}
\item[Tested with Version:] Debian 7.0 and PostgreSQL 9.1

\item[References:]

It's recommended to read 

{\small \url{http://www.postgresql.org/docs/X.X/interactive/runtime-config-connection.html\#RUNTIME-CONFIG-CONNECTION-SECURITY}}
(please change X.X with your preferred version e.g. 9.1).

\item[Settings:] \mbox{}


To start in SSL mode the server.crt and server.key must exist in the server's data directory \$PGDATA. 

Starting with version 9.2, you have the possibility to set the path.

\begin{lstlisting}[breaklines]
ssl_key_file = '/your/path/server.key'
ssl_cert_file = '/your/path/server.crt'
ssl_ca_file = '/your/path/root.crt'
\end{lstlisting}

\paragraph*{postgresql.conf}\mbox{}\\

\begin{lstlisting}[breaklines]
#>=8.3
ssl = on 
ssl_ciphers = 'EECDH+aRSA+AESGCM:EECDH+aRSA+SHA384:EECDH+aRSA+SHA256:EDH+CAMELLIA256:EECDH:EDH+aRSA:+SSLv3:!aNULL:!eNULL:!LOW:!3DES:!MD5:!EXP:!PSK:!SRP:!DSS:!RC4:!SEED:!AES128:!CAMELLIA128:!ECDSA:AES256-SHA'
\end{lstlisting}



\item[How to test:]
To test your ssl settings, run psql with the sslmode parameter:
\begin{lstlisting}[breaklines]
psql "sslmode=require host=postgres-server dbname=database" your-username
\end{lstlisting}

\end{description}

