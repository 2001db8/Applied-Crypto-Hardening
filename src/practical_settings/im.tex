% hack.
\gdef\currentsectionname{IM}
%%---------------------------------------------------------------------- 
\subsection{General server configuration recommendations}

For servers, we mostly recommend to apply what's proposed by the \emph{Peter's manifesto}\footnote{\url{https://github.com/stpeter/manifesto}}.

In short:
\begin{itemize*}
    \item require the use of TLS for both client-to-server and server-to-server connections
    \item prefer or require TLS cipher suites that enable forward secrecy
    \item deploy certificates issued by well-known and widely-deployed certification authorities (CAs)
\end{itemize*}

The last point being out-of-scope for this section, we will only cover the first two points.

\subsection{ejabberd}

\subsubsection{Tested with Versions}
\begin{itemize*}
  \item ejabberd 14.12, Debian 7 Wheezy
  \item ejabberd 14.12, Ubuntu 14.04 Trusty
  \item ejabberd 15.03, Ubuntu 14.04 Trusty
  \item ejabberd 16.01, Ubuntu 14.04 Trusty
\end{itemize*}

\subsubsection{Settings}
ejabberd is one of the popular Jabber servers. In order to be compliant
with the manifesto, you should adapt your configuration\footnote{\url{http://www.process-one.net/docs/ejabberd/guide_en.html}}:

\configfile{14.12/ejabberd.yml}{104-107,113-114,119-119,123-123,125-125,127-127,134-135,138-140,144-146,195-195,203-203,207-207,210-213}{%
  TLS setup for ejabberd}
% 
\subsubsection{Additional settings}
It is possible to explicitly specify a cipher string for TLS connections.
\configfile{14.12/ejabberd.yml}{104-107,113-114,119-119,123-123,125-125,127-127,134-135,138-138,142-142,144-146,195-195,203-203,207-207,210-213,217-217}{Specifying a cipher order and enforcing it}

Note that we are setting the SSL option \texttt{cipher\_server\_preference}. This enforces our cipher order when negotiating which ciphers are used, as the cipher order of some clients chooses weak ciphers over stronger ciphers.\footnote{\url{https://blog.thijsalkema.de/me/blog//blog/2013/09/02/the-state-of-tls-on-xmpp-3/}}

Starting with version 15.03\footnote{Early versions seem to have a few bugs - although officially supported, it did not work in tests with version 15.06. Version 16.01 is confirmed to work.}, it is possible to use custom Diffie-Hellman-Parameters. This allows us to negotiate stronger Diffie-Hellman-keys, and also helps us avoid problems with using common Diffie-Hellman-Parameters.\footnote{\url{https://weakdh.org}} You can generate your own parameter file by running:
\begin{lstlisting}
openssl dhparam -out dhparams.pem 4096
\end{lstlisting}

By default, ejabberd provides an administration website (look for the ejabberd\_http module). Enable TLS protection for it like this:

\configfile{14.12/ejabberd.yml}{177-178,181-183,185-185,187-188}{Adding TLS to the web interface}

\subsubsection{Tested with Versions}
\begin{itemize*}
\item Debian Wheezy 2.1.10-4+deb7u1
\end{itemize*}
\subsubsection{Settings}
Older versions of ejabberd use a different configuration file syntax.
In order to be compliant with the manifesto, you should adapt your configuration\footnote{\url{http://www.process-one.net/docs/ejabberd/guide_en.html}} as follows:
\configfile{2.1.10/ejabberd.cfg}{108-109,111-111,120-126,172-172,179-179,184-184}{
TLS setup for ejabberd}
%
\subsubsection{Additional settings}
Older versions of ejabberd ($ < $ 2.0.0) need to be patched\footnote{\url{http://hyperstruct.net/2007/06/20/installing-the-startcom-ssl-certificate-in-ejabberd/}} to be able to parse all of the certificates in the CA chain.
Specifying a custom cipher string is only possible starting with version 13.12 (see configuration for version 14.12 above).

\subsubsection{References}

\begin{itemize}
\item \href{http://www.process-one.net/en/ejabberd/docs/}{The ejabberd documentation: http://www.process-one.net/en/ejabberd/docs/}
\end{itemize}


\subsubsection{How to test}
\begin{itemize*}
  \item \url{https://xmpp.net} is a useful website to test Jabber server configurations.
\end{itemize*}


%%---------------------------------------------------------------------- 
\subsection{Chat privacy - Off-the-Record Messaging (OTR)}

The OTR protocol works on top of the Jabber protocol\footnote{\url{https://otr.cypherpunks.ca/Protocol-v3-4.0.0.html}}.  
It adds to popular chat clients (Adium, Pidgin...) the following properties for encrypted chats:
\begin{itemize*}
  \item Authentication
  \item Integrity
  \item Confidentiality
  \item Forward secrecy
\end{itemize*}

It basically uses Diffie-Hellman, AES and SHA1. Communicating over an insecure instant messaging network, OTR can be used for end to end encryption.

There are no specific configurations required but the protocol itself is worth to be mentioned.


%%---------------------------------------------------------------------- 
\subsection{Charybdis}
There are numerous implementations of IRC servers.  In this section, we choose \emph{Charybdis} which serves as basis for \emph{ircd-seven}\footnote{\url{https://dev.freenode.net/redmine/projects/ircd-seven}}, developed and used by freenode. Freenode is actually the biggest IRC network\footnote{\url{http://irc.netsplit.de/networks/top10.php}}. \emph{Charybdis} is part of the \emph{Debian} \& \emph{Ubuntu} distributions.
\configfile{ircd.conf}{12-12,15-15,24-24,44-44,60-60,63-63,66-66,74-75,83-83,132-132,138-138,144-144}{SSL relevant configuration for Charybdis/ircd-seven}


%%---------------------------------------------------------------------- 
\subsection{SILC}

SILC\footnote{\url{http://www.silcnet.org/} and
\url{https://en.wikipedia.org/wiki/SILC_(protocol)}} is instant messaging
protocol publicly released in 2000. SILC is a per-default secure chat protocol
thanks to a generalized usage of symmetric encryption. Keys are generated by
the server meaning that if compromised, communication could be compromised.

The protocol is not really popular anymore.

