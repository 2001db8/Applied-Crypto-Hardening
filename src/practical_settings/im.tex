%%\subsection{Instant Messaging Systems}
\subsubsection{XMPP / Jabber}
\todo{ts: Describe ejabberd configuration. Reference to Peter`s manifesto https://github.com/stpeter/manifesto}

\subsubsection{Server configuration}

For servers, we mostly recommend to apply what's proposed by the {\it Peter's manifesto}\footnote{https://github.com/stpeter/manifesto}.

In short:
\begin{itemize}
    \item require the use of TLS for both client-to-server and server-to-server connections
    \item prefer or require TLS cipher suites that enable forward secrecy
    \item deploy certificates issued by well-known and widely-deployed certification authorities (CAs)
\end{itemize}

The last point being out-of-scope for this section, we will only cover the first two points.

\paragraph{ejabberd}

ejabberd is one of the popular Jabber servers.  In order to be compliant
with the manifesto, you should adapt your
configuration\footnote{\url{http://www.process-one.net/docs/ejabberd/guide_en.html}}:
\begin{lstlisting}[breaklines]
{listen,
 [
  {5222, ejabberd_c2s, [
                        {access, c2s},
                        {shaper, c2s_shaper},
                        {max_stanza_size, 65536},
                        starttls,
                        starttls_required, 
                        {certfile, "/etc/ejabberd/ejabberd.pem"}
                       ]},
  {5269, ejabberd_s2s_in, [
                           {shaper, s2s_shaper},
                           {max_stanza_size, 131072}
                          ]},

  %%% Other input ports
]}.
{s2s_use_starttls, required_trusted}.
{s2s_certfile, "/etc/ejabberd/ejabberd.pem"}.
\end{lstlisting}


\subsubsection{Chat privacy - Off-the-Record Messaging (OTR)}

The OTR protocol works on top of the Jabber protocol(\footnote{https://otr.cypherpunks.ca/Protocol-v3-4.0.0.html}).  
It adds to popular chat clients (Adium, Pidgin...) the following propoerties for ciphered chats:
\begin{itemize}
    \item Authentification
    \item Integrity
    \item Confidentiality
    \item Forward privacy
\end{itemize}

It basically uses Diffie-Hellman, AES and SHA1. 

There are no specific configurations required but the protocol itself is worth to be mentioned.

\subsubsection{IRC}
