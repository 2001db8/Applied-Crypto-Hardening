% hack.
\gdef\currentsectionname{GPG}
\gdef\currentsubsectionname{GnuPG}

The OpenPGP protocol
\footnote{\url{https://tools.ietf.org/search/rfc4880}}
 uses asymmetric encryption to protect a session key which is used to encrypt a message. Additionally, it signs messages via asymmetric encryption and hash functions. %% what? - azet
Research on SHA-1 conducted back in 2005\footnote{\url{https://www.schneier.com/blog/archives/2005/02/sha1\_broken.html}} has made clear that collision attacks are a real threat to the security of the SHA-1 hash function. PGP settings should be adapted to avoid using SHA-1.


When using PGP, there are a couple of things to take care of:
\begin{itemize*}
  \item keylengths (see section \ref{section:keylengths})
  \item randomness (see section \ref{section:RNGs})
  \item preference of symmetric encryption algorithm (see section \ref{section:CipherSuites})
  \item preference of hash function (see section \ref{section:CipherSuites})
\end{itemize*}

Properly dealing with key material, passphrases and the web-of-trust is outside of the scope of this document. The GnuPG website\footnote{\url{http://www.gnupg.org/}} has a good tutorial on PGP.

This \href{https://www.debian-administration.org/users/dkg/weblog/48}{Debian How-to}\footnote{\url{https://www.debian-administration.org/users/dkg/weblog/48}} is a great resource on upgrading your old PGP key as well as on safe default settings. This section is built based on the Debian How-to.

\subsubsection{Hashing}
Avoid SHA-1 in GnuPG. Edit \$HOME/.gnupg/gpg.conf:

\configfile{gpg.conf}{208-210}{Digest selection in GnuPG}

Before you generate a new PGP key, make sure there is enough entropy available (see subsection \ref{subsec:RNG-linux}).

%\subsubsection{PGP / GPG Operations}

%% Ciphering - Unciphering operations
%%% TOO COMPLEX. Make a pointer to a good GPG tutorial

%% Signing / checking signatures
%%% TOO COMPLEX. Make a pointer to a good GPG tutorial

%\subsubsection{Trusted Keys}

%%Explain that a key by himself is not trustable.  Chain of trust principle.

%%% TOO COMPLEX. Make a pointer to a good GPG tutorial

%\subsection{Available implementations and mails plugins}

%% Microsoft Windows (Symantec for Outlook? GnuPG + ....)
%%% TOO COMPLEX. Make a pointer to a good GPG tutorial

%% Linux (GnuPG + Enigmail for Thunderbird)

%%% TOO COMPLEX. Make a pointer to a good GPG tutorial
%% Mac OS X (GnuPG + GPGMail)
%%% TOO COMPLEX. Make a pointer to a good GPG tutorial


