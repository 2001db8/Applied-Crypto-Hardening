\section{Cipher suites}

Cipher suites are a combination of algorithms to provide for 
Confidentiality, Integrity and Authenticity
\footnote{\url{http://en.wikipedia.org/wiki/Information\_security}} of 
communication. For example: sending encrypted data over the wire does not 
ensure that the data can not be modified (message integrity), similarly
encrypted data can be sent from an adversary. It is therefore paramount to
prove that data has been sent from the desired source (message authenticity).
This concept is known as authenticated encryption
\footnote{\url{http://en.wikipedia.org/wiki/Authenticated\_encryption}}
\footnote{\url{http://www.cs.jhu.edu/~astubble/dss/ae.pdf}}.

\subsection{Forward Secrecy}
Forward Secrecy or Perfect Forward Secrecy is a property of a cipher suite 
that ensures confidentiality even if the server key has been compromised.
Thus if traffic has been recorded it can not be decrypted even if an adversary
has got hold of the decryption key
\footnote{\url{http://en.wikipedia.org/wiki/Forward\_secrecy}}
\footnote{\url{https://www.eff.org/deeplinks/2013/08/pushing-perfect-forward-secrecy-important-web-privacy-protection}}. 

\subsection{Recommended cipher suites}

In principle, system administrators who want to improve their servers need to
make a hard decision between locking out some users while keeping very high
cipher suite security levels or supporting as many users as possible while
lowering some settings. \url{https://www.ssllabs.com/} gives administrators a
tool to test out different settings. The authors used ssllabs.com to arrive at
a set of cipher suites which we will recommend throught this document.
\textbf{Caution: these settings can only represent a subjective choice of the
authors at the time of this writing. It might be a wise choice to select your
own cipher suites based on the instructions in section
\ref{section:ChosingYourOwnCipherSuites}}.


\subsubsection{Configuration A: strong ciphers, fewer clients}

At the time of this writing, we recommend the following set of strong cipher
suites which may be useful in an environment where you do not depend on many,
diverse external clients and where compatibility is not an issue.  An example
of such an environment might be machine 2 machine communications or corporate
environments where you can define the software which must be used.


We arrived at this set of cipher suites by selecting

\begin{itemize}
\item TLS 1.2
\item Perfect forward secrecy / ephemeral Diffie Hellman
\item strong Hashes (SHA-2)
\item GCM as chaining mode if possible 
\end{itemize}

This results in the string:

\begin{lstlisting}[breaklines]
'EECDH+aRSA+AES256:EDH+aRSA+AES256:!SSLv3'
\end{lstlisting}

%$\implies$ resolves to 
%
%\begin{verbatim}
%openssl ciphers -V $string
%\end{verbatim}



\begin{center}

\begin{tabular}{lllllll}
\toprule
\textbf{ID}   & \textbf{OpenSSL Name}       & \textbf{Version} & \textbf{KeyEx} & \textbf{Auth} & \textbf{Cipher} & \textbf{Hash}\\\cmidrule(lr){1-7}
\verb|0xC030| & ECDHE-RSA-AES256-GCM-SHA384 & TLSv1.2          & ECDH           &  RSA          & AESGCM(256)     & AEAD         \\
\verb|0xC028| & ECDHE-RSA-AES256-SHA384     & TLSv1.2          & ECDH           &  RSA          & AES(256)        & SHA384       \\
\verb|0x009F| & DHE-RSA-AES256-GCM-SHA384   & TLSv1.2          & DH             &  RSA          & AESGCM(256)     & AEAD         \\
\verb|0x006B| & DHE-RSA-AES256-SHA256       & TLSv1.2          & DH             &  RSA          & AES(256)        & SHA256       \\
\bottomrule
\end{tabular}
\end{center}


\textbf{Compatibility}

Only clients which support TLS1.2 are covered by these cipher suites (Chrome 30,
Win 7 and Win 8.1 crypto stack, Opera 17, OpenSSL $\ge$ 1.0.1e, Safari 6 / iOS
6.0.1, Safari 7 / OS X 10.9).



\subsubsection{Configuration B: weaker ciphers, many clients}

In this section we propose a slighly "weaker" set of cipher suites. There are
some known weaknesses of for example SHA-1 which is included in this set.
However, the advantage of this set of cipher suites is its wider compatibility
with clients. 


\textbf{In the following document, all further examples in this paper will use Configuration B}.


We arrived at this set of cipher suites by selecting

\begin{itemize}
\item TLS 1.2, TLS 1.1, TLS 1.0
\item allowing SHA-1
\todo{AK: Note that SHA1 is considered broken but if we are in DHE, we might get around it as long as you can not calculate a SHA1 collision ``live'' on the wire}

\end{itemize}

This results in the string:

\begin{lstlisting}[breaklines]
'EECDH+aRSA+AESGCM:EECDH+aRSA+SHA384:EECDH+aRSA+SHA256:EDH+CAMELLIA256:EECDH:EDH+aRSA:+SSLv3:!aNULL:!eNULL:!LOW:!3DES:!MD5:!EXP:!PSK:!SRP:!DSS:!RC4:!SEED:!AES128:!CAMELLIA128:!ECDSA:AES256-SHA'
\end{lstlisting}



\begin{center}
\begin{tabular}{lllllll}
\toprule
\textbf{ID}   & \textbf{OpenSSL Name}       & \textbf{Version} & \textbf{KeyEx} & \textbf{Auth} & \textbf{Cipher} & \textbf{Hash}\\\cmidrule(lr){1-7}
\verb|0xC030| & ECDHE-RSA-AES256-GCM-SHA384 & TLSv1.2          & ECDH           &  RSA          & AESGCM(256)     & AEAD         \\ 
\verb|0xC028| & ECDHE-RSA-AES256-SHA384     & TLSv1.2          & ECDH           &  RSA          & AES(256)        & SHA384       \\ 
\verb|0x009F| & DHE-RSA-AES256-GCM-SHA384   & TLSv1.2          & DH             &  RSA          & AESGCM(256)     & AEAD         \\ 
\verb|0x006B| & DHE-RSA-AES256-SHA256       & TLSv1.2          & DH             &  RSA          & AES(256)        & SHA256       \\ 
\verb|0x0088| & DHE-RSA-CAMELLIA256-SHA     & SSLv3            & DH             &  RSA          & Camellia(256)   & SHA1         \\ 
\verb|0xC014| & ECDHE-RSA-AES256-SHA        & SSLv3            & ECDH           &  RSA          & AES(256)        & SHA1         \\ 
\verb|0x0039| & DHE-RSA-AES256-SHA          & SSLv3            & DH             &  RSA          & AES(256)        & SHA1         \\ 
\verb|0x0035| & AES256-SHA                  & SSLv3            & RSA            &  RSA          & AES(256)        & SHA1         \\
\bottomrule
\end{tabular}
\end{center}

\textbf{Compatibility}

Note that these cipher suites will not work with anything using Windows XP's
crypto stack (IE, Outlook), Java 6, Java 7 and Android 2.3. Java 7 could be
made compatible by installing the "Java Cryptography Extension (JCE) Unlimited
Strength Jurisdiction Policy Files"
(JCE) \footnote{\url{http://www.oracle.com/technetwork/java/javase/downloads/jce-7-download-432124.html}}.
We could not verify yet if installing JCE also fixes the Java 7
DH-parameter length limitation (1024 bit). 

\textbf{Explanation}

For a detailed explanation of the cipher suites chosen, please see
\ref{section:ChoosingYourOwnCipherSuites}. In short, finding the perfect cipher
string is impossible and must be a tradeoff. On the one hand
there are mandatory and optional ciphers defined in a few RFCs, on the other hand
there are clients and servers only implementing subsets of the specification.

Straight forward, we wanted strong ciphers, forward secrecy
\footnote{\url{http://nmav.gnutls.org/2011/12/price-to-pay-for-perfect-forward.html}}
and the most clients we could get while still having a cipher string that can be
used on older servers too (think OpenSSL 0.9.8). This cipher string is meant to be used
by copy and paste and needs to just work.

\begin{itemize}
\item TLS1.2 is preferred over TLSv1.0/SSLv3 (while still providing a useable cipher
      string for SSLv3).
\item AES256 and CAMELLIA256 count as strong ciphers at the moment; preferrably in
      GCM mode.\\
	  \todo{add a reference here please}
      \todo{Adi: add 128bit ciphers too} \\
      \todo{Team: discuss ordering of keys (256 $\rightarrow$ 128 or vice versa?)}
\item DHE or ECDHE for forward secrecy
\item RSA as this will fit most of todays setup
\item AES256-SHA as a last ressort (with this cipher at the end, even systems with
      very old versions of openssl like 0.9.8 will just work. Just forward secrecy
      will not be used. On systems that do not support elliptic curves, that cipher
      offers support for the Microsoft crypto libraries that only support ECDHE.
\end{itemize}
\todo{Adi: review "justification" when next section is written}



\subsection{Known insecure and weak cipher suites}
\todo{PG: please write this section. List all known broken, obsolete, weak and insecure cipher suites . Or even better: find the best site which keeps track of outdated cipher suites and simply reference it. We do not want to maintain such a list ourselves!}

\subsection{Compatibility}
\todo{write this section. The idea here is to first document which server (and openssl) version we assumed. Once these parameters are fixed, we then list all clients which are supported for Variant A) and B). Therefore we can document compatibilities to some extent. The sysadmin can then choose roughly what he looses or gains by omitting certain cipher suites.}


\subsection{Choosing your own cipher suites}
\label{section:ChoosingYourOwnCipherSuites}

\todo{ Adi...  you want to describe how to make your own selection of cipher suites here.}

SSL/TLS cipher suites consist of a key exchange mechanism, an authentication, a
stream cipher (or a block cipher with a chaining mode) and a message authentication
mechanism.

Many of those mechanisms are interchangeable like the key exchange in this example:
\texttt{ECDHE-RSA-AES256-GCM-SHA384} and \texttt{DHE-RSA-AES256-GCM-SHA384}.
To provide a decent level of security, all algorithms need to be safe (subject to
the disclaimer in section \ref{section:disclaimer}).

Note: There are some very weak cipher suites in about every crypto library, most of
them for historic reasons like the crypto export embargo
\footnote{\url{http://en.wikipedia.org/wiki/Export_of_cryptography_in_the_United_States}}.
For the following chapter support of those is assumed to be disabled by having
\texttt{!EXP:!LOW:!NULL} as part of the cipher string.

\todo{Adi: add boxes for summaries like here \\ \url{http://tex.stackexchange.com/questions/99809/box-or-sidebar-for-additional-text}}

\todo{Team: do we need references for all cipher suites considered weak?}

\subsubsection{key exchange}

RSA, DSA, DH, EDH, ECDSA, ECDH, EECDH, FORTEZZA(?).

Note that Elliptic Curves in current TLS standards (up to and including TLSv1.2) are
questioned by many experts due to the lack of documentation on how the parameters
specifying those curves were chosen. Without the EC mechanisms many clients, especially
those using the Windows crypto libraries will not be able to use Forward Secrecy as these
libraries only implement the elliptic curve variant of ephemeral DH key exchange. The
curves in question are SECP256, SECP384, SECP521 and SECP571 of which the first two
are the only ones implemented in the Windows crypto stack and they're the only listed
as a requirement in the RFC. %%TODO: add link to RFC

Other key exchange mechanisms like Pre-Shared Key (PSK) or Secure Remote Password
(SRP) are irrelevant for regular SSL/TLS use. \texttt{!PSK:!SRP}

\subsubsection{authentication}

RSA, DSA, DSS, ECDSA, ECDH, FORTEZZA(?).

Other authentication mechanisms like Pre Shared Keys aren't used in SSL/TLS: \texttt{!PSK:!aNULL}

\subsubsection{encryption}

AES, CAMELLIA, SEED, ARIA(?), FORTEZZA(?)...

Other ciphers like IDEA, RC2, RC4, 3DES or DES are weak and therefor not recommended:
\texttt{!DES:!3DES:!RC2:!RC4:!eNULL}

\subsubsection{message authentication}

SHA-1 (SHA), SHA-2 (SHA256, SHA384), AEAD

Note that SHA-1 is considered broken and should not be used. SHA-1 is however a the
only still available message authentication mechanism supporting TLS1.0/SSLv3. Without
SHA-1 most clients will be locked out.

Other hash functions like MD2, MD4 or MD5 are unsafe and broken: \texttt{!MD2:!MD4:!MD5}

\subsubsection{combining cipher strings}
%% reference 'man ciphers' and 'openssl ciphers' and show some simple examples
%% VERY IMPORTANT: hint at the IANA-list and the differences in implementations

\todo{ Adi...  The text below was simply the old text, still left here for reference.}

%%% NOTE: we do not need to list this all here, can move to an appendix
%At the time of this writing, SSL is defined in RFCs: 	
%
%\begin{itemize}
%\item RFC2246 - TLS1.0		
%\item RFC3268 - AES		
%\item RFC4132 - Camelia		
%\item RFC4162 - SEED		
%\item RFC4279 - PSK		
%\item RFC4346 - TLS 1.1		
%\item RFC4492 - ECC		
%\item RFC4785 - PSK\_NULL		
%\item RFC5246 - TLS 1.2		
%\item RFC5288 - AES\_GCM		
%\item RFC5289 - AES\_GCM\_SHA2\_ECC		
%\item RFC5430 - Suite B		
%\item RFC5487 - GCM\_PSK		
%\item RFC5489 - ECDHE\_PSK		
%\item RFC5932 - Camelia		
%\item RFC6101 - SSL 3.0		
%\item RFC6209 - ARIA		
%\item RFC6367 - Camelia		
%\item RFC6655 - AES\_CCM		
%\item RFC7027 - Brainpool Curves		
%\end{itemize}

\subsubsection{Overview of SSL Server settings}


Most Server software (Webservers, Mail servers, etc.) can be configured to prefer certain cipher suites over others. 
We followed the recommendations by Ivan Ristic's SSL/TLS Deployment Best Practices\footnote{\url{https://www.ssllabs.com/projects/best-practices/index.html}} document (see section 2.2 "Use Secure Protocols") and arrived at a list of recommended cipher suites for SSL enabled servers.

Following Ivan Ristic's adivce we arrived at a categorisation of cipher suites.

\begin{center}
\begin{tabular}{lllll}
\cmidrule[\heavyrulewidth]{2-5}
& \textbf{Version}   & \textbf{KeyEx} & \textbf{Cipher}    & \textbf{MAC}       \\\cmidrule(lr){2-5}
\cellcolor{green}prefer  & TLS 1.2   & DHE\_DSS   & AES\_256\_GCM   & SHA384        \\
    &   & DHE\_RSA   & AES\_256\_CCM   & SHA256        \\
    &   & ECDHE\_ECDSA   & AES\_256\_CBC   &       \\
    &   & ECDHE\_RSA &   &       \\ 
    &   &   &   &       \\
\cellcolor{orange}consider    & TLS 1.1   & DH\_DSS    & AES\_128\_GCM   & SHA       \\
    & TLS 1.0   & DH\_RSA    & AES\_128\_CCM   &       \\
    &   & ECDH\_ECDSA    & AES\_128\_CBC   &       \\ 
    &   & ECDH\_RSA  & CAMELLIA\_256\_CBC  &       \\
    &   & RSA   & CAMELLIA\_128\_CBC  &       \\
    &   &   &   &       \\
\cellcolor{red}avoid   
& SSL 3.0   & NULL  & NULL  & NULL      \\
    &   & DH\_anon   & RC4\_128   & MD5       \\
    &   & ECDH\_anon & 3DES\_EDE\_CBC  &       \\
    &   &   & DES\_CBC   &       \\
    &   &   &   &       \\
\cellcolor{blue}{\color{white}special }
&   & PSK   & CAMELLIA\_256\_GCM  &       \\
    &   & DHE\_PSK   & CAMELLIA\_128\_GCM  &       \\
    &   & RSA\_PSK   & ARIA\_256\_GCM  &       \\
    &   & ECDHE\_PSK & ARIA\_256\_CBC  &       \\
    &   &   & ARIA\_128\_GCM  &       \\
    &   &   & ARIA\_128\_CBC  &       \\
    &   &   & SEED  &       \\
\cmidrule[\heavyrulewidth]{2-5}
\end{tabular}
\end{center}

A remark on the ``consider'' section: the BSI (Federal office for information security, Germany) recommends in its technical report TR-02102-2\footnote{\url{https://www.bsi.bund.de/SharedDocs/Downloads/DE/BSI/Publikationen/TechnischeRichtlinien/TR02102/BSI-TR-02102-2_pdf.html}} to \textbf{avoid} non-ephemeral\footnote{Ephemeral keys are session keys which are destroyed upon termination of the encrypted session. In TLS/SSL, they are realized by the DHE cipher suites. } keys for any communication which might contain personal or sensitive data. In this document, we follow BSI's advice and therefore only keep cipher suites containing (EC)DH\textbf{E} (ephemeral) variants. System administrators, who can not use forward secrecy can still use the cipher suites in the ``consider'' section. We however, do not recommend them in this document.

%% NOTE: s/forward secrecy/perfect forward secrecy???

Note that the entries marked as ``special'' are cipher suites which are not common to all clients (webbrowsers etc).


\subsubsection{Tested clients}
 
Next we tested the cipher suites above on the following clients:

%% NOTE: we need to test with more systems!!
\begin{itemize}
\item Chrome 30.0.1599.101 Mac OS X 10.9
\item Safari 7.0 Mac OS X 10.9
\item Firefox 25.0 Mac OS X 10.9
\item Internet Explorer 10 Windows 7
\item Apple iOS 7.0.3
\end{itemize}


The result of testing the cipher suites with these clients gives us a preference order as shown in table \ref{table:prefOrderCipherSuites}. 
Should a client not be able to use a specific cipher suite, it will fall back to the next possible entry as given by the ordering.

\begin{table}[h]
\centering\small
    \begin{tabular}{cllcccc}
    \toprule
    \textbf{Pref}   & \textbf{Cipher Suite}                            & \textbf{ID}   & \multicolumn{4}{l}{\textbf{Supported by}}\\ 
    \cmidrule(lr){4-7}
                    & \textbf{OpenSSL Name}                            &               & Chrome & FF   & IE   & Safari \\
    \cmidrule(lr){1-7}
    \phantom{0}1    & \verb|TLS_DHE_RSA_WITH_AES_256_GCM_SHA384|     & \verb|0x009f| & \no    & \no  & \no  & \no    \\
                    & \verb|DHE-RSA-AES256-GCM-SHA384|                      &               & &&&\\\rowcolor{lightgray}
    \phantom{0}2    & \verb|TLS_ECDHE_ECDSA_WITH_AES_256_CBC_SHA384| & \verb|0xC024| & \no    & \no  & \no  & \yes   \\\rowcolor{lightgray}
                    & \verb|ECDHE-ECDSA-AES256-SHA384|                      &               & &&&\\
    \phantom{0}3    & \verb|TLS_ECDHE_RSA_WITH_AES_256_CBC_SHA384|   & \verb|0xC028| & \no    & \no  & \no  & \yes   \\
                    & \verb|ECDHE-RSA-AES256-SHA384|                        &               & &&&\\\rowcolor{lightgray}
    \phantom{0}4    & \verb|TLS_DHE_RSA_WITH_AES_256_CBC_SHA256|     & \verb|0x006B| & \yes   & \no  & \no  & \yes   \\\rowcolor{lightgray}
                    & \verb|DHE-RSA-AES256-SHA256|                          &               & &&&\\
    \phantom{0}5    & \verb|TLS_ECDHE_ECDSA_WITH_AES_256_CBC_SHA|    & \verb|0xC00A| & \yes   & \yes & \yes & \yes   \\
                    & \verb|ECDHE-ECDSA-AES256-SHA|                         &               & &&&\\\rowcolor{lightgray}
    \phantom{0}6    & \verb|TLS_ECDHE_RSA_WITH_AES_256_CBC_SHA|      & \verb|0xC014| & \yes   & \yes & \yes & \yes   \\\rowcolor{lightgray}
                    & \verb|ECDHE-RSA-AES256-SHA|                           &               & &&&\\
    \phantom{0}7    & \verb|TLS_DHE_RSA_WITH_AES_256_CBC_SHA|        & \verb|0x0039| & \yes   & \yes & \no  & \yes   \\
                    & \verb|DHE-RSA-AES256-SHA|                             &               & &&&\\\rowcolor{lightgray}
    \phantom{0}8    & \verb|TLS_DHE_DSS_WITH_AES_256_CBC_SHA|        & \verb|0x0038| & \no    & \yes & \yes & \no    \\\rowcolor{lightgray}
                    & \verb|DHE-DSS-AES256-SHA|                             &               & &&&\\
    \phantom{0}9    & \verb|TLS_DHE_RSA_WITH_CAMELLIA_256_CBC_SHA|   & \verb|0x0088| & \no    & \yes & \no  & \no    \\
                    & \verb|DHE-RSA-CAMELLIA256-SHA|                        &               & &&&\\\rowcolor{lightgray}
    \phantom{}10    & \verb|TLS_DHE_DSS_WITH_CAMELLIA_256_CBC_SHA|   & \verb|0x0087| & \no    & \yes & \no  & \no    \\\rowcolor{lightgray}
                    & \verb|DHE-DSS-CAMELLIA256-SHA|                        &               & &&&\\
   \bottomrule
    \end{tabular}
\caption{Preference order of cipher suites.  All suites are supported by OpenSSL.}
\label{table:prefOrderCipherSuites}
\end{table}

Note: the above table \ref{table:prefOrderCipherSuites} contains Elliptic curve key exchanges. There are currently strong doubts\footnote{\url{http://safecurves.cr.yp.to/rigid.html}} concerning ECC.
If unsure, remove the cipher suites starting with ECDHE in the table above.


Based on this ordering, we can now define the corresponding settings for servers. We will start with the most common web servers.

