\section{Public Key Infrastructures}
\label{section:PKIs}

Public-Key Infrastructures aim to provide a way to simplify the verification of
a certificates trustworthiness.  For this, certificate authorities (CAs) are
used to create a signature chain from the root CA down to the server (or client).
Accepting a CA as a generally-trusted mediator solves the trust-scaling problem
at the cost of introducing an actor that magically is more trustworthy.

This section deals with settings related to trusting CAs. However, our main
recommendations for PKIs is: if you are able to run your own PKI and disable
any other CA, do so. This makes sense most in environments where any machine-to-machine
communication system compatibility with external entities is not an issue.
%% azet:
%% this needs discussion! self-signed certificates simply do not work in practices
%% for real-world scenarios - i.e. websites that actually serve a lot of people

A good background on PKIs can be found in 
\footnote{\url{https://developer.mozilla.org/en/docs/Introduction_to_Public-Key_Cryptography}}
\footnote{\url{http://cacr.uwaterloo.ca/hac/about/chap8.pdf}}
\footnote{\url{http://www.verisign.com.au/repository/tutorial/cryptography/intro1.shtml}}
.

\todo{ts: Background and Configuration (EMET) of Certificate Pinning, TLSA integration, 
  When to use self-signed certificates, how to get certificates from public CA authorities 
  (CACert, StartSSL), Best-practices how to create a CA and how to generate private keys/CSRs, 
  Discussion about OCSP and CRLs. TD: Useful Firefox plugins: CipherFox, Conspiracy, Perspectives.}


%``Certification
%Policy''\footnote{\url{http://en.wikipedia.org/wiki/Certificate_Policy}}
%(CA)
