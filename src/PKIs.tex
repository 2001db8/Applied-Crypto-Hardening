\section{Public Key Infrastructures}
\label{section:PKIs}

Public-Key Infrastructures aim to provide a way to simplify the verification of
a certificate's trustworthiness.  For this, certificate authorities (CAs) are
used to create a signature chain from the root CA down to the server (or client).
Accepting a CA as a generally-trusted mediator solves the trust-scaling problem
at the cost of introducing an actor that magically is more trustworthy.

In the first part of this section an overview is given, how to get a certificate from
a trusted CA, or how to setup your own CA. In the second part recommendations will be 
given how you can improve the security of PKIs.

\section{Certificate Authorities}
\label{sec:cas}
In order to get a certificate, you either go to a CA which will issue a certificate for you,
or you run your own CA. In the latter case one normally speaks of self-signed 
certificates. For both cases there are pros and contras and the decision is
as always security versus usability.

\subsubsection{Signed Certificates form a trusted Certificate Authority}
\label{sec:signcertfromca}
Trusted certificates are mostly issued by commercial CAs, such as Verisign, GoDaddy, Teletrust, etc.
These certificates are mostly issued for a specified time period and cost money. Nevertheless
there are also free trusted CAs available, such as StartSSL, or CACert.

If you consider getting a signed certificate from a trusted CA, you should not let the CA generate the 
private key for you. You can generate a private key and a corresponding certificate request as follows:

\begin{lstlisting}[breaklines]
% openssl req -new -nodes -keyout <servername>.key -out <servername>.csr -newkey rsa:<keysize>
Country Name (2 letter code) [AU]:DE
State or Province Name (full name) [Some-State]:Bavaria
Locality Name (eg, city) []:Munich
Organization Name (eg, company) [Internet Widgits Pty Ltd]:Example
Organizational Unit Name (eg, section) []:Example Section
Common Name (e.g. server FQDN or YOUR name) []:example.com
Email Address []:admin@example.com

Please enter the following 'extra' attributes
to be sent with your certificate request
A challenge password []:
An optional company name []:
\end{lstlisting}

\subsubsection{Setting up your own Certificate Authority}
\label{sec:setupownca}
In some situations it is sufficient to use your own certificate authority. An example is an OpenVPN
installation or if only internal systems communicate with each other, without interaction with users. 
Creating a CA can be accomplished using OpenSSL (Debian):

\begin{lstlisting}
% cd /usr/lib/ssl/misc
% sudo ./CA.pl -newca
\end{lstlisting}

Answer the questions according to your setup. Now you have configured your basic settings and 
issued a new root certificate. Now you can issue new certificates as follows:

\begin{lstlisting}
% cd /usr/lib/ssl/misc
% sudo ./CA.pl -newreq
\end{lstlisting}

\subsection{Hardening PKI}
\label{sec:hardeningpki}
In recent years several CAs were compromised by attackers in order to
get ahold of trusted certificates for malicious activities. In 2011 
the Dutch CA Diginotar was hacked and all certificates were
revoked~\cite{diginotar-hack}. Recently Google found certificates
issued to them, which were not used by the
company~\cite{googlecahack}. The concept of PKIs heavily depends on the
security of CAs.  If they get compromised the whole PKI system will
fail. Some CAs tend to incorrectly issue certificates that were designated
to do a different job than what they were intended to by the CA~\cite{gocode}.

Therefore several security enhancements were introduced by different
organisations and vendors~\cite{tschofenig-webpki}. Currently two
methods are used, DANE~\cite{rfc6698} and Certificate
Pinning~\cite{draft-ietf-websec-key-pinning}. Google recently proposed
a new system to detect malicous CAs and certificates  called Certificate 
Transparency~\cite{certtransparency}.

% \subsubsection{DANE}
% \label{sec:dane}

% \subsubsection{Certificate Pinning}
% \label{sec:certpinning}



% This section deals with settings related to trusting CAs. However,
% our main recommendations for PKIs is: if you are able to run your
% own PKI and disable any other CA, do so. This makes sense most in
% environments where any machine-to-machine communication system
% compatibility with external entities is not an issue.
%% azet: this needs discussion! self-signed certificates simply do not
%% work in practices for real-world scenarios - i.e. websites that
%% actually serve a lot of people

% A good background on PKIs can be found in
% \footnote{\url{https://developer.mozilla.org/en/docs/Introduction_to_Public-Key_Cryptography}}
% \footnote{\url{http://cacr.uwaterloo.ca/hac/about/chap8.pdf}}
% \footnote{\url{http://www.verisign.com.au/repository/tutorial/cryptography/intro1.shtml}}
% .

% \todo{ts: Background and Configuration (EMET) of Certificate Pinning,
%   TLSA integration, When to use self-signed certificates, how to get
%   certificates from public CA authorities (CACert, StartSSL),
%   Best-practices how to create a CA and how to generate private
%   keys/CSRs, Discussion about OCSP and CRLs. TD: Useful Firefox
%   plugins: CipherFox, Conspiracy, Perspectives.}


% ``Certificate Policy''\cite{Wikipedia:Certificate_Policy} (CA)
