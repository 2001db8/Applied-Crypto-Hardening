\section{Public Key Infrastructures}
\label{section:PKIs}

Public-Key Infrastructures aim to provide a way to simplify the verification of
a certificate's trustworthiness.  For this, certificate authorities (CAs) are
used for creating a signature chain from the CA down to the server (or client).
Accepting a CA as a generally-trusted mediator solves the trust-scaling problem
at the cost of introducing an actor that magically is more trustworthy.

This section deals with settings related to trusting CAs.  However, our main
recommendations for PKIs is: if you are able to run your own PKI and disable
any other CA, do so.  This is mostly possible in any machine 2 machine
communication system where compatibility with externalities is not an issue.

A good background on PKIs can be found in \todo{insert reference}.

\todo{ts: Background and Configuration (EMET) of Certificate Pinning, TLSA integration, 
  When to use self-signed certificates, how to get certificates from public CA authorities 
  (CACert, StartSSL), Best-practices how to create a CA and how to generate private keys/CSRs, 
  Discussion about OCSP and CRLs. TD: Useful Firefox plugins: CipherFox, Conspiracy, Perspectives.}


%``Certification
%Policy''\footnote{\url{http://en.wikipedia.org/wiki/Certificate_Policy}}
%(CA)
