\section{Methods}

Since many years, NIST\footnote{\url{http://www.nist.gov/}} is the most
prominent standardisation institute which industry would consult for
recommendations in the field of cryptography. However, the NSA leaks of 2013
showed that even certain NIST recommendations were
subverted\footnote{\url{http://www.scientificamerican.com/article.cfm?id=nsa-nist-encryption-scandal}}
by the NSA.  As a consequence, NIST initiated a review process of their
standardisation
efforts\footnote{\url{http://csrc.nist.gov/groups/ST/crypto-review/index.html}}.
For the purposes of this document and at the time of this writing, we
can not blindly trust NIST's recommendations on cipher and cipher suite
settings at this very moment. 

Instead, we chose to collect the most well known facts about crypto-settings
and let as many trusted specialists as possible review these settings.  The
review process is completely open and done on a public mailing list. The
document is available (read-only) to the public Internet on a git server and
open for public scrutiny. However, write permissions to the document are only
granted to trusted people. The list of editors is made public.  Every write
operation to the document is logged via the ``git'' version control system and
thus can be traced back to a specific author.  We do not trust an unknown git
server. 

Public peer-review / ``multiple eyes'' checking our recommendation is the best
strategy we can imagine at the moment.

C.O.S.H.E.R. = completely open source, headers, engineering and research!
