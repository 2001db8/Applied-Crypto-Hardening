 % This file is public domain
 % If you want to use arara, you need the following directives:
 % arara: pdflatex: { synctex: on }
 % arara: makeglossaries
 % arara: pdflatex: { synctex: on }
 % arara: makeglossaries
 % arara: pdflatex: { synctex: on }
\documentclass{report}

\usepackage[colorlinks]{hyperref}
\usepackage[toc,index,nohypertypes={index}]{glossaries}
\usepackage{glossary-mcols}

\makeglossaries

 % define terms for the index

\newterm[plural={stegosauruses}]{stegosaurus}
\newterm[plural={triceratopses}]{triceratops}
\newterm[plural={apatosauruses}]{apatosaurus}

 % To avoid labels conflicting with the same name in the main glossary
 % prefix the index label with "ind-". (This means that the name must
 % be set independently.)
\newterm[name={dinosaur}]{ind-dinosaur}
\newterm[name={Triassic}]{ind-triassic}

 % define terms for the main glossary

\newglossaryentry{dinosaur}%
{%
  name={dinosaur\glsadd{ind-dinosaur}},%
  description={One of a group of dinosauria}%
}

\newglossaryentry{Triassic}%
{%
  name={Triassic\glsadd{ind-triassic}},%
  description={The first period of the Mesozoic Era}
}

\begin{document}
\tableofcontents

\chapter{Dinosaurs}

\Glspl{dinosaur} are now extinct. They first appeared during the
\gls{Triassic} period. Examples of \glspl{dinosaur} include the
\gls{triceratops}, the \gls{apatosaurus} and the \gls{stegosaurus}.

Indexed term: \gls{ind-dinosaur}.

\renewcommand*{\glsnamefont}[1]{\textbf{\makefirstuc{#1}}}

\printglossary[style=long,nogroupskip]

\renewcommand*{\glsnamefont}[1]{\textmd{#1}}

\printindex[style=mcolindexgroup]

\end{document}
