\documentclass{nlctdoc}

\usepackage{alltt}
\usepackage{mfirstuc}
\usepackage{pifont}
\usepackage[utf8]{inputenc}
\ifpdf
\usepackage[T1]{fontenc}
\fi
\usepackage{cmap}
\usepackage[colorlinks,
            bookmarks,
            hyperindex=false,
            pdfauthor={Nicola L.C. Talbot},
            pdftitle={mfirstuc.sty: uppercasing first letter},
            pdfkeywords={LaTeX,package,uppercase}]{hyperref}

\begin{document}
 \title{mfirstuc.sty v1.08: uppercasing first letter}
 \author{Nicola L.C. Talbot\\[10pt]
Dickimaw Books\\
\url{http://www.dickimaw-books.com/}}

 \date{2013-11-04}
 \maketitle

The \styfmt{glossaries} bundle is supplied with the package 
\sty{mfirstuc} which provides the command:
\begin{definition}[\DescribeMacro{\makefirstuc}]
\cs{makefirstuc}\marg{stuff}
\end{definition}
This makes the first object of
\meta{stuff} uppercase unless \meta{stuff} starts with a control
sequence followed by a non-empty group, in which case the first
object in the group is converted to uppercase. Examples:
\begin{itemize}
\item |\makefirstuc{abc}| produces \makefirstuc{abc}.

\item |\makefirstuc{\emph{abc}}| produces \makefirstuc{\emph{abc}}
(\ics{MakeUppercase} has been applied to the letter \qt{a} rather
than \cs{emph}). Note however that
\begin{verbatim}
\makefirstuc{{\em abc}}
\end{verbatim}
produces \makefirstuc{{\em abc}} (first object is |{\em abc}| so
equivalent to |\MakeUppercase{\em abc}|), and
\begin{verbatim}
{\makefirstuc{\em abc}}
\end{verbatim}
produces {\makefirstuc{\em abc}} (|\em| doesn't have an argument
therefore first object is |\em| so equivalent to
|{\MakeUppercase{\em}abc}|).

\item |\makefirstuc{{\'a}bc}| produces \makefirstuc{{\'a}bc}.

\item |\makefirstuc{\ae bc}| produces \makefirstuc{\ae bc}.

\item |\makefirstuc{{\ae}bc}| produces \makefirstuc{{\ae}bc}.

\item |\makefirstuc{{ä}bc}| produces \makefirstuc{{ä}bc}.

\end{itemize}
Note that non-Latin or accented characters appearing at the
start of the text must be placed in a group (even if you are
using the \sty{inputenc} package) due to expansion issues.

\begin{important}
In version 1.02 of \styfmt{mfirstuc}, a bug fix resulted in a change
in output if the first object is a control sequence followed by
an empty group. Prior to version 1.02, |\makefirstuc{\ae{}bc}|
produced \ae Bc. However as from version 1.02, it now produces
\AE bc.
\end{important}

Note also that
\begin{verbatim}
\newcommand{\abc}{abc}
\makefirstuc{\abc}
\end{verbatim}
produces: ABC. This is because the first object in the argument of
\cs{makefirstuc} is \cs{abc}, so it does |\MakeUppercase{\abc}|.
Whereas:
\begin{verbatim}
\newcommand{\abc}{abc}
\expandafter\makefirstuc\expandafter{\abc}
\end{verbatim}
produces: Abc. There is a short cut command which will do this:
\begin{definition}[\DescribeMacro{\xmakefirstuc}]
\cs{xmakefirstuc}\marg{stuff}
\end{definition}
This is equivalent to \cs{expandafter}\cs{makefirstuc}\cs{expandafter}\marg{stuff}. So
\begin{verbatim}
\newcommand{\abc}{abc}
\xmakefirstuc{\abc}
\end{verbatim}
produces: 
\newcommand{\abc}{abc}%
\xmakefirstuc{\abc}.

\begin{important}
\cs{xmakefirstuc} only performs one level expansion on the first
object in its argument. It does not fully expand the entire
argument.
\end{important}

If you use \styfmt{mfirstuc} without the \sty{glossaries} package, 
the standard \cs{MakeUppercase} command is used. If used with
\sty{glossaries}, \ics{MakeTextUppercase} (defined by \sty{textcase}
the package) is used instead. If you are using \styfmt{mfirstuc}
without the \styfmt{glossaries} package and want to use
\ics{MakeTextUppercase} instead, you can redefine
\begin{definition}[\DescribeMacro\glsmakefirstuc]
\cs{glsmakefirstuc}\marg{text}
\end{definition}
For example:
\begin{verbatim}
\renewcommand{\glsmakefirstuc}[1]{\MakeTextUppercase #1}
\end{verbatim}
Remember to also load \sty{textcase} (\styfmt{glossaries} loads this
automatically).

New to mfirstuc v1.06:
\begin{definition}[\DescribeMacro{\capitalisewords}]
\cs{capitalisewords}\marg{text}
\end{definition}
This command apply \ics{makefirstuc} to each word in \meta{text}
where the space character is used as the word separator. Note that
it has to be a plain space character, not another form of space,
such as \verb|~| or \cs{space}. Note that no expansion is performed
on \meta{text}.

\begin{definition}[\DescribeMacro{\xcapitalisewords}]
\cs{xcapitalisewords}\marg{text}
\end{definition}
This is a short cut for
\cs{expandafter}\cs{capitalisewords}\cs{expandafter}\marg{text}.

\begin{important}
If you are using \sty{hyperref} and want to use \cs{capitalisewords}
or \ics{makefirstuc} (or \cs{xcapitalisewords}\slash\cs{xmakefirstuc}) 
in a section heading, the PDF bookmarks won't be able to use the command 
as it's not expandable, so you will get a warning that looks like:
\begin{verbatim}
Package hyperref Warning: Token not allowed in a PDF string (PDFDocEncoding):
(hyperref)                removing `\capitalisewords'
\end{verbatim}
If you want to provide an alternative for the PDF bookmark, you can
use \sty{hyperref}'s \ics{texorpdfstring} command. See the
\sty{hyperref} manual for further details.
\end{important}

Examples:
\begin{enumerate}
\item 
\begin{verbatim}
\capitalisewords{a book of rhyme.}
\end{verbatim}
produces:
\capitalisewords{a book of rhyme.}

\item
\begin{verbatim}
\capitalisewords{a book\space of rhyme.}
\end{verbatim}
produces:
\capitalisewords{a book\space of rhyme.}

\item
\begin{verbatim}
\newcommand{\mytitle}{a book\space of rhyme.}
\capitalisewords{\mytitle}
\end{verbatim}
produces:
\newcommand{\mytitle}{a book\space of rhyme.}
\capitalisewords{\mytitle}
(No expansion is performed on \cs{mytitle}, so \meta{text}
consists of just one ``word''.) Compare with next example:

\item
\begin{verbatim}
\newcommand{\mytitle}{a book\space of rhyme.}
\xcapitalisewords{\mytitle}
\end{verbatim}
produces:
\xcapitalisewords{\mytitle}

\end{enumerate}

\end{document}
