\chapter{Tools}
\label{cha:tools}
This section lists tools for checking the security settings.

\section{SSL \& TLS}

Server checks via the web
\begin{itemize*}
  \item \href{https://ssllabs.com}{ssllabs.com} offers a great way to check your webserver for misconfigurations. See \url{https://www.ssllabs.com/ssltest/}. Furthermore, ssllabs.com has a good best practices tutorial, which focuses on avoiding the most common mistakes in SSL.
  \item SSL Server certificate installation issues \url{https://www.sslshopper.com/ssl-checker.html}
  \item Check SPDY protocol support and basic TLS setup \url{http://spdycheck.org/}
  \item XMPP/Jabber Server check (Client-to-Server and Server-to-Server) \url{https://xmpp.net/}
  \item Luxsci SMTP TLS Checker \url{https://luxsci.com/extranet/tlschecker.html}
  \item Does your mail server support StartTLS? \url{https://starttls.info/}
  \item \url{http://checktls.com} is a tool for testing arbitrary TLS services.
  \item TLS and SSH key check \url{https://factorable.net/keycheck.html}
  \item \url{http://tls.secg.org} is a tool for testing interoperability of HTTPS implementations for ECC cipher suites.
  \item \url{http://www.whynopadlock.com/} Testing for mixed SSL parts loaded via http that can totally lever your HTTPS.
\end{itemize*}

Browser checks
\begin{itemize*}
  \item Check your browser's SSL capabilities: \url{https://cc.dcsec.uni-hannover.de/} and \url{https://www.ssllabs.com/ssltest/viewMyClient.html}.
  \item Check Browsers SSL/TLS support and vulnerability to attacks: \url{https://www.howsmyssl.com}
\end{itemize*}


Command line tools
\begin{itemize*}
  \item \url{https://sourceforge.net/projects/sslscan} connects to a given SSL service and shows the cipher suites that are offered.
  \item \url{http://www.bolet.org/TestSSLServer/} tests for BEAST and CRIME vulnerabilities.
  \item \url{https://github.com/drwetter/testssl.sh} checks a server's service on any port for the support of TLS/SSL ciphers, protocols as well as some cryptographic flaws (CRIME, BREACH, CCS, Heartbleed).
  \item \url{https://github.com/iSECPartners/sslyze} Fast and full-featured SSL scanner
  \item \url{https://github.com/jvehent/cipherscan} Fast TLS scanner (ciphers, order, protocols, key size and more)
  \item \url{http://nmap.org/} nmap security scanner
  \item \url{http://www.openssl.net} OpenSSL s\_client
\end{itemize*}


\section{Key length}
\begin{itemize*}
  \item \url{http://www.keylength.com} comprehensive online resource for comparison of key lengths according to common recommendations and standards in cryptography.
\end{itemize*}


\section{RNGs}

%% NOTE: should we merge that with chapter 6.6??
\begin{itemize*}
  \item \href{http://www.fourmilab.ch/random/}{ENT} is a pseudo random number generator sequence tester.
  \item \href{http://www.issihosts.com/haveged/}{HaveGE} is a tool which increases the Entropy of the Linux random number generator devices. It is based on the HAVEGE algorithm. \url{http://dl.acm.org/citation.cfm?id=945516}
  \item \href{http://www.phy.duke.edu/~rgb/General/dieharder.php}{Dieharder} a random number generator testing tool.
  \item \href{http://www.cacert.at/random/}{CAcert Random} another random number generator testing service.
\end{itemize*}

\section{Guides}
\begin{itemize*}
  \item See: \url{https://www.ssllabs.com/downloads/SSL_TLS_Deployment_Best_Practices_1.3.pdf}.
\end{itemize*}
