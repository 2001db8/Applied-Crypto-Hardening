\section{Tools}

This section lists tools for checking the security settings.

\subsection{SSL \& TLS}

Check your browser's ssl capabilities: \url{https://cc.dcsec.uni-hannover.de/}


ssllabs.com offers a great way to check your webserver for misconfigurations. See \url{https://www.ssllabs.com/ssltest/}.
Furthermore, ssllabs.com has a good best practices tutorial, which focuses on avoiding the most common mistakes in SSL.
See: \url{https://www.ssllabs.com/downloads/SSL_TLS_Deployment_Best_Practices_1.3.pdf}
%% this breaks my pdf converter hmm

\url{http://tls.secg.org} is a tool for testing interoperability of HTTPS implementations for ECC cipher suites.

\url{http://sourceforge.net/projects/sslscan} connects to a given SSL
service and shows the cipher suites that are offered.

\url{http://checktls.com} is a tool for testing arbitrary TLS services. 

\url{https://github.com/iSECPartners/sslyze} Fast and full-featured SSL scanner

\subsection{Keylength}

\url{http://www.keylength.com} comprehensive online resource for comparison of keylenghts according to common recommendatons and standards in cryptography.

\subsection{RNGs}

%% NOTE: should we merge that with chapter 6.6??
\begin{itemize}
\item \href{http://www.fourmilab.ch/random/}{ENT} is a pseudo random number generator sequence tester.  
\item \href{http://www.issihosts.com/haveged/}{HaveGE} is a tool which increases the Entropy of the Linux random number generator devices. It is based on the HAVEGE algorithm. \url{http://dl.acm.org/citation.cfm?id=945516}
\end{itemize}



