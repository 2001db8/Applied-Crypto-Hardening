\section{Scope}
\label{section:Scope}

In this guide, we restricted ourselves to:
\begin{itemize}
\item Internet-facing services
\item Commonly used services
\item Devices which are used in business environments (this specifically excludes XBoxes, Playstations and similar consumer devices)
\item OpenSSL 
\end{itemize}

We explicitly excluded:
\begin{itemize}
\item Specialized systems (such as medical devices, most embedded systems, etc.)
\item Wireless Access Points
\item Smart-cards/chip cards
%\item Services which should be run only in an internal network and never face the Internet.
\end{itemize}

%% * whatsapp --> man kann nichts machen, out of scope
%* Lync: == SIP von M$.
%* Skype: man kann ncihts machen, out of scope.
%* Wi-Fi APs, 802.1X, ... ???? --> out of scope
%* Tomcats/...????
%* SIP   -> Klaus???
%* SRTP  -> Klaus???
%* DNSSec ?? Verweis auf BCPxxx  --> out of scope
%   - DANE
%What happens at the IETF at the moment?
%* TOR?? --> out of scope
%* S/Mime --> nachsehen, gibt es BCPs? (--> Ramin)
%* TrueCrypt, LUKS, FileVault, etc ---> out of scope
%* AFS -> out of scope
%* Kerberos --> out of scope
%* NNTP -> out of scope
%* NTPs tlsdate -> out of scope
%* BGP / OSPF --> out of scope
%* irc,silc --> out of scope
%* LDAP -> out of scope
%* Moxa , APC, und co... ICS . Ethernet to serial --> out of scope
%* telnet -> DON't!!!
%* rsyslog --> out of scope
%* ARP bei v6 spoofing -> out of scope
%* tinc?? -> out of scope
%* rsync -> nur ueber ssh fahren ausser public web mirrors
%* telnets -> out of scope
%* ftps -> out of scope
%seclayer-tcp    3495/udp    # securitylayer over tcp
%seclayer-tcp    3495/tcp    # securitylayer over tcp
%* webmin -> maybe
%* plesk -> out of scope
%* phpmyadmin --> haengt am apache, out of scope
%* DSL modems -> out of scope
%* UPnP, natPmp --> out of scope 
