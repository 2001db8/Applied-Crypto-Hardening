\section*{Abstract}

This guide arose out of the need for system administrators to have an
updated, solid, well researched and thought-through guide for configuring SSL,
PGP, SSH and other cryptographic tools in the post-Snowden age. Triggered by the NSA
leaks in the summer of 2013, many system administrators and IT security
officers saw the need to strengthen their encryption settings.
This guide is specifically written for these system administrators.

\vskip 0.5em

As Schneier noted\cite{Sch13},
it seems that intelligence agencies and adversaries on the Internet are not
breaking so much the mathematics of encryption per se, but rather use software
and hardware weaknesses, subvert standardization processes, plant backdoors,
rig random number generators and most of all exploit careless settings in
server configurations and encryption systems to listen in on private
communications. 

\vskip 0.5em

This guide can only address one aspect of securing our information
systems: getting the crypto settings right to the best of the authors' current
knowledge. Other attacks, as the above mentioned, require different protection
schemes which are not covered in this guide. This guide is
not an introduction to cryptography on how to use PGP nor SSL. For
background information on cryptography, cryptoanalysis, PGP and SSL we would
like to refer the reader to the the chapters \ref{section:Tools},
\ref{section:Links} and \ref{section:Suggested_Reading} at the end of this
document.

\vskip 0.5em

The focus of this guide is merely to give current best practices for
configuring complex cipher suites and related parameters in a \textbf{copy \&
paste-able manner}. The guide tries to stay as concise as is possible for such
a complex topic as cryptography. There are many guides and best practice
documents available when it comes to cryptography. However none of them focuses
specifically on what an average system administrator needs for hardening his or
her systems' crypto settings.

This guide tries to fill this gap.



