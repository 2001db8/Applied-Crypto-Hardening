\section*{Abstract}

This whitepaper arose out of the need for system administrators to have an
updated, solid, well researched and thought-through guide for configuring SSL,
PGP, SSH and other cryptographic tools in the post-PRISM age.  Since the NSA
leaks in the summer of 2013, many system administrators and IT security
officers see the need to update their encryption settings.

However, as Schneier
noted\footnote{\url{https://www.schneier.com/blog/archives/2013/09/the\_nsa\_is\_brea.html}},
it seems that intelligence agencies and adversaries on the Internet are not
breaking so much the mathematics of encryption per se, but rather use
weaknesses and sloppy settings in encryption frameworks to break the codes,
next to using other means such as ``kinetic-decryption'' (breaking in, stealing
keys) or planting backdoors and rigging random number generators, etc.

This whitepaper can only address one aspect of securing our
information systems: getting the crypto settings right. Other attacks, as the
above mentioned, require different protection schemes which are not covered in
this whitepaper. 

This whitepaper is \textbf{not} an introduction to cryptography, on how to use
PGP nor SSL. Its focus is merely to give current best practices with
configuring complex cipher suites and related parameters in a \textbf{copy \&
paste-able manner}.  For background information on cryptography,
cryptoanalysis, PGP and SSL we would like to refer the reader to the list of
books at the end of this document.
