\section*{Abstract}

This whitepaper arose out of the need for system administrators to have an
updated, solid, well researched and thought-through guide for configuring SSL,
PGP, SSH and other cryptographic tools in the post-PRISM age.  Triggered by the NSA
leaks in the summer of 2013, many system administrators and IT security
officers saw the need to strengthen their encryption settings.
This guide is specifically written for these system administrators.

\vskip 0.5em

As Schneier
noted\footnote{\url{https://www.schneier.com/blog/archives/2013/09/the\_nsa\_is\_brea.html}},
it seems that intelligence agencies and adversaries on the Internet are not
breaking so much the mathematics of encryption per se, but rather use software
and hardware weaknesses, subvert standardization processes, plant backdoors,
rig random number generators and most of all exploit careless settings in
server configurations and encryption systems to listen in on private
communications. 

\vskip 0.5em

This whitepaper can only address one aspect of securing our information
systems: getting the crypto settings right. Other attacks, as the above
mentioned, require different protection schemes which are not covered in this
whitepaper.  This whitepaper is \textbf{not} an introduction to cryptography
on how to use PGP nor SSL.  For background information on cryptography,
cryptoanalysis, PGP and SSL we would like to refer the reader to the list of
books at the end of this document.

\vskip 0.5em

The focus of this guide is merely to give current best practices  for
configuring complex cipher suites and related parameters in a \textbf{copy \&
paste-able manner}.  The guide tries to stay as concise as is possible for such
a complex topic as cryptography.  There are many guides and best practice
documents available when it comes to cryptography, however none of them focus
on what a system administrator needs to do precisely for his system to harden
its security with respect to cipher suites. Therefore we focus on copy \&
paste-able settings.


