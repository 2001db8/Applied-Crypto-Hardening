\section{How to read this guide}
\label{sec:how-read-this}
This guide tries to accommodate two needs: first of all, having a handy reference on how to configure the most common services' crypto settings and second of all, explain a bit of background on cryptography. This background is essential if the reader wants to chose his or her own cipher string settings.

System administrators who want to copy \& paste recommendations quickly without
spending a lot of time on background reading on cryptography or cryptanalysis
can do so, by simply searching for the corresponding section in chapter
\ref{chapter:PracticalSettings} (``Practical recommendations''). 

It is important to know that in this guide the authors arrived at two recommendations: \textit{Cipher string A} and \textit{Cipher string B}. While the former is a hardened recommendation the latter is a weaker one but provides wider compatibility.
\textit{Cipher strings A and B} are described in \ref{section:recommendedciphers}. 

%\textit{Cipher
%string B} covers the most common use-cases (such as running an e-commerce shop,
%a private homepage, a mail server, $ \ldots $).

However, for the quick copy \& paste approach it is important to know that this
guide assumes users are happy with \textit{Cipher string B}. 


While chapter \ref{chapter:PracticalSettings} is intended to serve as a copy \& paste reference, chapter \ref{chapter:Theory} (``Theory'') explains the reasoning behind \textit{cipher string B}. In particular, section \ref{section:CipherSuites} explains how to choose individual cipher strings. We advise the reader to actually read this section and challenge our reasoning in choosing \textit{Cipher string B} and to come up with a better  or localized solution.

%We start with some general remarks in sections \ref{section:DH},\ref{section:EllipticCurveCryptography},\ref{section:keylengths} on 
%If you are a system administrator and want to quickly update your services, jump right to section \ref{section:PracticalSettings}. However, we recommend that you take some time and first read through the theory part (chapter \ref{chapter:Theory}), especially section \ref{section:CipherSuites} on how to choose your own cipher string and then adapt the settings in section \ref{section:PracticalSettings} to your own needs.

\tikzstyle{terminator} = [ellipse, draw,  minimum height=2em,
    text width=4.5em, text badly centered, inner sep=0pt]
\tikzstyle{decision} = [diamond, draw,aspect=2,
     text width=10em, text badly centered, node distance=8em, inner sep=0pt]
\tikzstyle{block} = [rectangle, draw,inner sep=0pt,
     text width=17em, text centered, rounded corners, minimum height=4em]
\tikzstyle{line} = [draw, very thick, -latex']
\tikzstyle{decision answer}=[near start,color=black]
\begin{tikzpicture}[scale=1, node distance = 6em, auto]
    % Place nodes
    \node [terminator] (start) {Start};
    \node [block, right of=start, text width=7em, node distance=8em] (intro) {%
      \nameref{chapter:Intro}};
    \node [decision, below of=intro] (evaluate) {%
      No time, I just want to copy \& paste};
    \node [block, right of=evaluate, node distance=20em] (practical1) {%
      read \nameref{chapter:PracticalSettings}};
    \node [block, below of=evaluate,node distance=8em ] (theory) {%
      To understand why we chose certain settings, read
      \nameref{chapter:Theory} first};
    \node [block, right of=theory, node distance=20em] (practical2) {%
      re-read \nameref{chapter:PracticalSettings}};
    \node [block, below of =practical2] (appendix) {%
      \hyperref[appendix]{Appendix}: references, links};
    % Draw edges
    \path [line] (start) -- (intro);
    \path [line] (intro) -- (evaluate);
    \path [line] (evaluate) -- node [decision answer]  {yes} (practical1);
    \path [line] (evaluate) -- node [decision answer]  {no} (theory);
    \path [line] (practical1) -- (theory);
    \path [line] (theory) -- (practical2);
    \path [line] (practical2) -- (appendix);
\end{tikzpicture}

%%% Local Variables: 
%%% mode: latex
%%% TeX-master: "applied-crypto-hardening"
%%% End: 
