\section{Recommendations on practical settings}


\subsection{SSL}

%%% NOTE: we do not need to list this all here, can move to an appendix
%At the time of this writing, SSL is defined in RFCs: 	
%
%\begin{itemize}
%\item RFC2246 - TLS1.0		
%\item RFC3268 - AES		
%\item RFC4132 - Camelia		
%\item RFC4162 - SEED		
%\item RFC4279 - PSK		
%\item RFC4346 - TLS 1.1		
%\item RFC4492 - ECC		
%\item RFC4785 - PSK\_NULL		
%\item RFC5246 - TLS 1.2		
%\item RFC5288 - AES\_GCM		
%\item RFC5289 - AES\_GCM\_SHA2\_ECC		
%\item RFC5430 - Suite B		
%\item RFC5487 - GCM\_PSK		
%\item RFC5489 - ECDHE\_PSK		
%\item RFC5932 - Camelia		
%\item RFC6101 - SSL 3.0		
%\item RFC6209 - ARIA		
%\item RFC6367 - Camelia		
%\item RFC6655 - AES\_CCM		
%\item RFC7027 - Brainpool Curves		
%\end{itemize}

\subsubsection{Overview of SSL Server settings}

Most Server software (Webservers, Mail servers, etc.) can be configured to prefer certain cipher suites over others. 
We followed the recommendations by Ivan Ristic's \cite{RisticSSLTLSDeploymentBestPractices}SSL/TLS Deployment Best Practices document (section 2.2 "Use Secure Protocols") and arrived at a list of recommended cipher suites for SSL enabled servers.

The results of following his adivce is a categorisation of cipher suites.

\begin{center}
\begin{tabular}{| l | l | l | l | l|}
\hline
& Version   & Key\_Exchange  & Cipher    & MAC       \\ \hline
\cellcolor{green}prefer  & TLS 1.2   & DHE\_DSS   & AES\_256\_GCM   & SHA384        \\ \hline
    &   & DHE\_RSA   & AES\_256\_CCM   & SHA256        \\ \hline
    &   & ECDHE\_ECDSA   & AES\_256\_CBC   &       \\ \hline
    &   & ECDHE\_RSA &   &       \\ \hline
    &   &   &   &       \\ \hline
\cellcolor{orange}consider    & TLS 1.1   & DH\_DSS    & AES\_128\_GCM   & SHA       \\ \hline
    & TLS 1.0   & DH\_RSA    & AES\_128\_CCM   &       \\ \hline
    &   & ECDH\_ECDSA    & AES\_128\_CBC   &       \\ \hline
    &   & ECDH\_RSA  & CAMELLIA\_256\_CBC  &       \\ \hline
    &   & RSA   & CAMELLIA\_128\_CBC  &       \\ \hline
    &   &   &   &       \\ \hline
\cellcolor{red}avoid   
& SSL 3.0   & NULL  & NULL  & NULL      \\ \hline
    &   & DH\_anon   & RC4\_128   & MD5       \\ \hline
    &   & ECDH\_anon & 3DES\_EDE\_CBC  &       \\ \hline
    &   &   & DES\_CBC   &       \\ \hline
    &   &   &   &       \\ \hline
\cellcolor{blue}{\color{white}special }
&   & PSK   & CAMELLIA\_256\_GCM  &       \\ \hline
    &   & DHE\_PSK   & CAMELLIA\_128\_GCM  &       \\ \hline
    &   & RSA\_PSK   & ARIA\_256\_GCM  &       \\ \hline
    &   & ECDHE\_PSK & ARIA\_256\_CBC  &       \\ \hline
    &   &   & ARIA\_128\_GCM  &       \\ \hline
    &   &   & ARIA\_128\_CBC  &       \\ \hline
    &   &   & SEED  &       \\ \hline
\end{tabular}
\end{center}


Note that the entries marked as "special" are cipher suites which are not common to all clients (webbrowsers etc).


\subsubsection{Clients}
 
Next we tested the cipher suites above on the following clients:

\begin{itemize}
\item Chrome 30.0.1599.101 Mac OS X 10.9
\item Safari 7.0 Mac OS X 10.9
\item Firefox 25.0 Mac OS X 10.9
\item Internet Explorer 10 Windows 7
\item Apple iOS 7.0.3
\end{itemize}


The result of testing the cipher suites with these clients gives us the following result and a preference order. 
Should a client not be able to use a specific cipher suite, it will fall back to the next possible entry as given by the ordering.

\begin{table}[h]
\small
    \begin{tabular}{|l|l|l|l|l|}
    \hline
    Pref & Cipher Suite                                   & ID         & Browser                     \\ \hline
    1    & TLS\_ECDHE\_ECDSA\_WITH\_AES\_256\_CBC\_SHA384 &     0xC024 & Safari                      \\ \hline
    2    & TLS\_ECDHE\_RSA\_WITH\_AES\_256\_CBC\_SHA384   &     0xC028 & Safari                      \\ \hline
    3    & TLS\_DHE\_RSA\_WITH\_AES\_256\_CBC\_SHA256     &     0x006B & Safari, Chrome              \\ \hline
    4    & TLS\_ECDHE\_ECDSA\_WITH\_AES\_256\_CBC\_SHA    &     0xC00A & Safari, Chrome, Firefox, IE \\ \hline
    5    & TLS\_ECDHE\_RSA\_WITH\_AES\_256\_CBC\_SHA      &     0xC014 & Safari, Chrome, Firefox, IE \\ \hline
    6    & TLS\_DHE\_RSA\_WITH\_AES\_256\_CBC\_SHA        &     0x0039 & Safari, Chrome, Firefox     \\ \hline
    7    & TLS\_DHE\_DSS\_WITH\_AES\_256\_CBC\_SHA        &     0x0038 & Firefox, IE                 \\ \hline
    8    & TLS\_DHE\_RSA\_WITH\_CAMELLIA\_256\_CBC\_SHA   &     0x0088 & Firefox                     \\ \hline
    9    & TLS\_DHE\_DSS\_WITH\_CAMELLIA\_256\_CBC\_SHA   &     0x0087 & Firefox                     \\ \hline
    \end{tabular}
\end{table}

\FloatBarrier

The same data again, specifying the OpenSSL name:

\begin{table}[h]
\small
\FloatBarrier
    \begin{tabular}{|l|l|l|}
    \hline
    Cipher Suite                                   & ID            & OpenSSL Name                  \\ \hline
    TLS\_ECDHE\_ECDSA\_WITH\_AES\_256\_CBC\_SHA384 &     0xC024 &     ECDHE-ECDSA-AES256-SHA384 \\ \hline
    TLS\_ECDHE\_RSA\_WITH\_AES\_256\_CBC\_SHA384   &     0xC028 &     ECDHE-RSA-AES256-SHA384   \\ \hline
    TLS\_DHE\_RSA\_WITH\_AES\_256\_CBC\_SHA256     &     0x006B &     DHE-RSA-AES256-SHA256     \\ \hline
    TLS\_ECDHE\_ECDSA\_WITH\_AES\_256\_CBC\_SHA    &     0xC00A &     ECDHE-ECDSA-AES256-SHA    \\ \hline
    TLS\_ECDHE\_RSA\_WITH\_AES\_256\_CBC\_SHA      &     0xC014 &     ECDHE-RSA-AES256-SHA      \\ \hline
    TLS\_DHE\_RSA\_WITH\_AES\_256\_CBC\_SHA        &     0x0039 &     DHE-RSA-AES256-SHA        \\ \hline
    TLS\_DHE\_DSS\_WITH\_AES\_256\_CBC\_SHA        &     0x0038 &     DHE-DSS-AES256-SHA        \\ \hline
    TLS\_DHE\_RSA\_WITH\_CAMELLIA\_256\_CBC\_SHA   &     0x0088 &     DHE-RSA-CAMELLIA256-SHA   \\ \hline
    TLS\_DHE\_DSS\_WITH\_CAMELLIA\_256\_CBC\_SHA   &     0x0087 &     DHE-DSS-CAMELLIA256-SHA   \\ \hline
    \end{tabular}
\end{table}



Based on this ordering, we can now define the corresponding settings for servers. We will start with the most common web servers

\subsubsection{Apache}

Note: a "\textbackslash" (backslash) denotes a line continuation which was wrapped due to formatting reasons here. Do not copy it verbatim.

\begin{verbatim}
  SSLProtocol ALL -SSLv2
  SSLHonorCipherOrder On
  SSLCipherSuite  ECDH+AESGCM:DH+AESGCM:\
    ECDHE-ECDSA-AES256-SHA384:ECDHE-RSA-AES256-SHA384:\
    DHE-RSA-AES256-SHA256:ECDHE-ECDSA-AES256-SHA:\
    ECDHE-RSA-AES256-SHA:DHE-RSA-AES256-SHA:DHE-DSS-AES256-SHA:\
    DHE-RSA-CAMELLIA256-SHA:DHE-DSS-CAMELLIA256-SHA:\
    !ADH:!AECDH:!MD5:!DSS
\end{verbatim}

%XXXX   ECDH+AES256:DH+AES256:ECDH+AES128:DH+AES:ECDH+3DES:DH+3DES:RSA+AES:RSA+3DES:!ADH:!AECDH:!MD5:!DSS



\subsubsection{nginx}


\subsubsection{openssl.conf settings}

%\subsubsection{Differences in SSL libraries: gnutls vs. openssl vs. others}

\subsubsection{IMAPS}
\subsubsection{SMTP: opportunistic TLS}
% do we need to documment starttls in detail?
%\subsubsection{starttls?}

\subsection{SSH}

\subsection{OpenVPN}

\subsection{PGP}

\subsection{PRNG settings}
