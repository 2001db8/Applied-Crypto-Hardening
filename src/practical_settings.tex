\section{Recommendations on practical settings}


\subsection{SSL}

%%% NOTE: we do not need to list this all here, can move to an appendix
%At the time of this writing, SSL is defined in RFCs: 	
%
%\begin{itemize}
%\item RFC2246 - TLS1.0		
%\item RFC3268 - AES		
%\item RFC4132 - Camelia		
%\item RFC4162 - SEED		
%\item RFC4279 - PSK		
%\item RFC4346 - TLS 1.1		
%\item RFC4492 - ECC		
%\item RFC4785 - PSK\_NULL		
%\item RFC5246 - TLS 1.2		
%\item RFC5288 - AES\_GCM		
%\item RFC5289 - AES\_GCM\_SHA2\_ECC		
%\item RFC5430 - Suite B		
%\item RFC5487 - GCM\_PSK		
%\item RFC5489 - ECDHE\_PSK		
%\item RFC5932 - Camelia		
%\item RFC6101 - SSL 3.0		
%\item RFC6209 - ARIA		
%\item RFC6367 - Camelia		
%\item RFC6655 - AES\_CCM		
%\item RFC7027 - Brainpool Curves		
%\end{itemize}

\subsubsection{Overview of SSL Server settings}

Most Server software (Webservers, Mail servers, etc.) can be configured to prefer certain cipher suites over others. 
We followed the recommendations by Ivan Ristic's SSL/TLS Deployment Best Practices\footnote{\url{https://www.ssllabs.com/projects/best-practices/index.html}} document (see section 2.2 "Use Secure Protocols") and arrived at a list of recommended cipher suites for SSL enabled servers.

Following Ivan Ristic's adivce we arrived at a categorisation of cipher suites.

\begin{center}
\begin{tabular}{| l | l | l | l | l|}
\hline
& Version   & Key\_Exchange  & Cipher    & MAC       \\ \hline
\cellcolor{green}prefer  & TLS 1.2   & DHE\_DSS   & AES\_256\_GCM   & SHA384        \\ \hline
    &   & DHE\_RSA   & AES\_256\_CCM   & SHA256        \\ \hline
    &   & ECDHE\_ECDSA   & AES\_256\_CBC   &       \\ \hline
    &   & ECDHE\_RSA &   &       \\ \hline
    &   &   &   &       \\ \hline
\cellcolor{orange}consider    & TLS 1.1   & DH\_DSS    & AES\_128\_GCM   & SHA       \\ \hline
    & TLS 1.0   & DH\_RSA    & AES\_128\_CCM   &       \\ \hline
    &   & ECDH\_ECDSA    & AES\_128\_CBC   &       \\ \hline
    &   & ECDH\_RSA  & CAMELLIA\_256\_CBC  &       \\ \hline
    &   & RSA   & CAMELLIA\_128\_CBC  &       \\ \hline
    &   &   &   &       \\ \hline
\cellcolor{red}avoid   
& SSL 3.0   & NULL  & NULL  & NULL      \\ \hline
    &   & DH\_anon   & RC4\_128   & MD5       \\ \hline
    &   & ECDH\_anon & 3DES\_EDE\_CBC  &       \\ \hline
    &   &   & DES\_CBC   &       \\ \hline
    &   &   &   &       \\ \hline
\cellcolor{blue}{\color{white}special }
&   & PSK   & CAMELLIA\_256\_GCM  &       \\ \hline
    &   & DHE\_PSK   & CAMELLIA\_128\_GCM  &       \\ \hline
    &   & RSA\_PSK   & ARIA\_256\_GCM  &       \\ \hline
    &   & ECDHE\_PSK & ARIA\_256\_CBC  &       \\ \hline
    &   &   & ARIA\_128\_GCM  &       \\ \hline
    &   &   & ARIA\_128\_CBC  &       \\ \hline
    &   &   & SEED  &       \\ \hline
\end{tabular}
\end{center}

A remark on the ``consider'' section: the BSI (Federal office for information security, Germany) recommends in its technical report TR-02102-2\footnote{\url{https://www.bsi.bund.de/SharedDocs/Downloads/DE/BSI/Publikationen/TechnischeRichtlinien/TR02102/BSI-TR-02102-2_pdf.html}} to \textbf{avoid} non-ephemeral\footnote{ephemeral keys are session keys which are destroyed upon termination of the encrypted session. In TLS/SSL, they are realized by the DHE cipher suites. } keys for any communication which might contain personal or sensitive data. In this document, we follow BSI's advice and therefore only keep cipher suites containing (EC)DH\textbf{E} (ephemeral) variants. System administrators, who can not use forward secrecy can still use the cipher suites in the ``consider'' section. We however, do not recommend them in this document.

%% NOTE: s/forward secrecy/perfect forward secrecy???

Note that the entries marked as ``special'' are cipher suites which are not common to all clients (webbrowsers etc).


\subsubsection{Tested clients}
 
Next we tested the cipher suites above on the following clients:

%% NOTE: we need to test with more systems!!
\begin{itemize}
\item Chrome 30.0.1599.101 Mac OS X 10.9
\item Safari 7.0 Mac OS X 10.9
\item Firefox 25.0 Mac OS X 10.9
\item Internet Explorer 10 Windows 7
\item Apple iOS 7.0.3
\end{itemize}


The result of testing the cipher suites with these clients gives us a preference order as shown in table \ref{table:prefOrderCipherSuites}. 
Should a client not be able to use a specific cipher suite, it will fall back to the next possible entry as given by the ordering.

\begin{center}
\begin{table}[h]
\small
    \begin{tabular}{|l|l|l|l|l|}
    \hline
    Pref & Cipher Suite                                   & ID         & Browser                     \\ \hline
    1    & TLS\_DHE\_RSA\_WITH\_AES\_256\_GCM\_SHA384 	  &  	0x009f & OpenSSL command line client \\ \hline
    2    & TLS\_ECDHE\_ECDSA\_WITH\_AES\_256\_CBC\_SHA384 &     0xC024 & Safari                      \\ \hline
    3    & TLS\_ECDHE\_RSA\_WITH\_AES\_256\_CBC\_SHA384   &     0xC028 & Safari                      \\ \hline
    4    & TLS\_DHE\_RSA\_WITH\_AES\_256\_CBC\_SHA256     &     0x006B & Safari, Chrome              \\ \hline
    5    & TLS\_ECDHE\_ECDSA\_WITH\_AES\_256\_CBC\_SHA    &     0xC00A & Safari, Chrome, Firefox, IE \\ \hline
    6    & TLS\_ECDHE\_RSA\_WITH\_AES\_256\_CBC\_SHA      &     0xC014 & Safari, Chrome, Firefox, IE \\ \hline
    7    & TLS\_DHE\_RSA\_WITH\_AES\_256\_CBC\_SHA        &     0x0039 & Safari, Chrome, Firefox     \\ \hline
    8    & TLS\_DHE\_DSS\_WITH\_AES\_256\_CBC\_SHA        &     0x0038 & Firefox, IE                 \\ \hline
    9    & TLS\_DHE\_RSA\_WITH\_CAMELLIA\_256\_CBC\_SHA   &     0x0088 & Firefox                     \\ \hline
    10   & TLS\_DHE\_DSS\_WITH\_CAMELLIA\_256\_CBC\_SHA   &     0x0087 & Firefox                     \\ \hline
    \end{tabular}
\caption{Preference order of cipher suites}
\label{table:prefOrderCipherSuites}
\end{table}
\end{center}


Table \ref{table:prefOrderOpenSSLNames} shows the same data again with specifying the corresponding OpenSSL name.

\begin{center}
\begin{table}[h]
\small
    \begin{tabular}{|l|l|l|}
    \hline
    Cipher Suite                                   & ID         & OpenSSL Name                  \\ \hline
    TLS\_DHE\_RSA\_WITH\_AES\_256\_GCM\_SHA384     &  	 0x009f & 	  DHE-RSA-AES256-GCM-SHA384 \\ \hline
    TLS\_ECDHE\_ECDSA\_WITH\_AES\_256\_CBC\_SHA384 &     0xC024 &     ECDHE-ECDSA-AES256-SHA384 \\ \hline
    TLS\_ECDHE\_RSA\_WITH\_AES\_256\_CBC\_SHA384   &     0xC028 &     ECDHE-RSA-AES256-SHA384   \\ \hline
    TLS\_DHE\_RSA\_WITH\_AES\_256\_CBC\_SHA256     &     0x006B &     DHE-RSA-AES256-SHA256     \\ \hline
    TLS\_ECDHE\_ECDSA\_WITH\_AES\_256\_CBC\_SHA    &     0xC00A &     ECDHE-ECDSA-AES256-SHA    \\ \hline
    TLS\_ECDHE\_RSA\_WITH\_AES\_256\_CBC\_SHA      &     0xC014 &     ECDHE-RSA-AES256-SHA      \\ \hline
    TLS\_DHE\_RSA\_WITH\_AES\_256\_CBC\_SHA        &     0x0039 &     DHE-RSA-AES256-SHA        \\ \hline
    TLS\_DHE\_DSS\_WITH\_AES\_256\_CBC\_SHA        &     0x0038 &     DHE-DSS-AES256-SHA        \\ \hline
    TLS\_DHE\_RSA\_WITH\_CAMELLIA\_256\_CBC\_SHA   &     0x0088 &     DHE-RSA-CAMELLIA256-SHA   \\ \hline
    TLS\_DHE\_DSS\_WITH\_CAMELLIA\_256\_CBC\_SHA   &     0x0087 &     DHE-DSS-CAMELLIA256-SHA   \\ \hline
    \end{tabular}
\caption{Preference order of cipher suites, with OpenSSL names}
\label{table:prefOrderOpenSSLNames}
\end{table}
\end{center}

Note: the tables \ref{table:prefOrderOpenSSLNames} and \ref{table:prefOrderCipherSuites} contain Eliptic curve key exchanges. There are currently strong doubts\footnote{\url{http://safecurves.cr.yp.to/rigid.html}} concerning ECC.
If unsure, remove the cipher suites starting with ECDHE in the table above.


Based on this ordering, we can now define the corresponding settings for servers. We will start with the most common web servers

\subsubsection{Apache}

Note: a "\textbackslash" (backslash) denotes a line continuation which was wrapped due to formatting reasons here. Do not copy it verbatim.

%-All +TLSv1.1 +TLSv1.2
\begin{verbatim}
  SSLProtocol All -SSLv2 -SSLv3 
  SSLHonorCipherOrder On
  SSLCompression off
  # Add six earth month HSTS header for all users...
  Header add Strict-Transport-Security "max-age=15768000"
  # If you want to protect all subdomains, use the following header
  # ALL subdomains HAVE TO support https if you use this!
  # Strict-Transport-Security: max-age=15768000 ; includeSubDomains

  SSLCipherSuite  DHE+AESGCM:\
    ECDHE-ECDSA-AES256-SHA384:ECDHE-RSA-AES256-SHA384:\
    DHE-RSA-AES256-SHA256:ECDHE-ECDSA-AES256-SHA:\
    ECDHE-RSA-AES256-SHA:DHE-RSA-AES256-SHA:\
    DHE-DSS-AES256-SHA:DHE-RSA-CAMELLIA256-SHA:\
    DHE-DSS-CAMELLIA256-SHA:!ADH:!AECDH:!MD5:!DSS
\end{verbatim}

Note again, that any cipher suite starting with ECDHE  can be omitted in case of doubt.
%% XXX NOTE TO SELF: remove from future automatically generated lists!

You should redirect everything to httpS:// if possible. In Apache you can do this with the following setting inside of a VirtualHost environment:

\begin{verbatim}
  <VirtualHost *:80>
   #...
   RewriteEngine On
        RewriteRule ^.*$ https://%{SERVER_NAME}%{REQUEST_URI} [L,R=permanent]
   #...
  </VirtualHost>
\end{verbatim}

%XXXX   ECDH+AES256:DH+AES256:ECDH+AES128:DH+AES:ECDH+3DES:DH+3DES:RSA+AES:RSA+3DES:!ADH:!AECDH:!MD5:!DSS



\subsubsection{nginx}

\begin{verbatim}
  ssl_prefer_server_ciphers on;
  ssl_protocols -SSLv2 -SSLv3; 
  ssl_ciphers DHE+AESGCM:\
    ECDHE-ECDSA-AES256-SHA384:ECDHE-RSA-AES256-SHA384:\
    DHE-RSA-AES256-SHA256:ECDHE-ECDSA-AES256-SHA:\
    ECDHE-RSA-AES256-SHA:DHE-RSA-AES256-SHA:\
    DHE-DSS-AES256-SHA:DHE-RSA-CAMELLIA256-SHA:\
    DHE-DSS-CAMELLIA256-SHA:!ADH:!AECDH:!MD5:!DSS;
  add_header Strict-Transport-Security max-age=2592000;
  add_header X-Frame-Options DENY;
\end{verbatim}

%% XXX FIXME: do we need to specify dhparams? Parameter: ssl_dhparam = file. See: http://wiki.nginx.org/HttpSslModule#ssl_protocols


If you decide to trust NIST's ECC curve recommendation, you can add the following line to nginx's configuration file to select special curves:

\begin{verbatim}
  ssl_ecdh_curve          sect571k1;
\end{verbatim}

You should redirect everything to httpS:// if possible. In Nginx you can do this with the following setting:

\begin{verbatim}
  rewrite     ^(.*)   https://$host$1 permanent;
\end{verbatim}

%\subsubsection{openssl.conf settings}

%\subsubsection{Differences in SSL libraries: gnutls vs. openssl vs. others}

\subsubsection{Dovecot}

Dovecot 2.2:

% Example: http://dovecot.org/list/dovecot/2013-October/092999.html

\begin{verbatim}
  ssl_cipher_list = DHE+AESGCM:\
    ECDHE-ECDSA-AES256-SHA384:ECDHE-RSA-AES256-SHA384:\
    DHE-RSA-AES256-SHA256:ECDHE-ECDSA-AES256-SHA:\
    ECDHE-RSA-AES256-SHA:DHE-RSA-AES256-SHA:\
    DHE-DSS-AES256-SHA:DHE-RSA-CAMELLIA256-SHA:\
    DHE-DSS-CAMELLIA256-SHA:!ADH:!AECDH:!MD5:!DSS
  ssl_prefer_server_ciphers = yes
\end{verbatim}

Dovecot 2.1: Almost as good as dovecot 2.2. Does not support ssl\_prefer\_server\_ciphers


\subsubsection{Cyrus}

\subsubsection{UW}

Another option to secure IMAPs servers is to place them behind an stunnel server. 

% XXX config von Adi?
% sslVersion = TLSv1
% ciphers = EDH+CAMELLIA256:EDH+aRSA:+SSLv3:!aNULL:!eNULL:!LOW:!3DES:!MD5:!EXP:!PSK:!SRP:!DSS:!RC4:!SEED:-AES128:!CAMELLIA128:!ECDSA:AES256-SHA:EDH+AES128;
% options = CIPHER_SERVER_PREFERENCE
% TIMEOUTclose = 1

\subsubsection{Postfix}

First, you need to generate Diffie Hellman parameters (please first take a look at the section \ref{section:PRNG}):

\begin{verbatim}
  % openssl gendh -out /etc/postfix/dh_param_512.pem -2 512
  % openssl gendh -out /etc/postfix/dh_param_1024.pem -2 1024
\end{verbatim}

Next, we specify these DH parameters in the postfix config file:

\begin{verbatim}
  smtpd_tls_dh512_param_file = /etc/postfix/dh_param_512.pem
  smtpd_tls_dh1024_param_file = /etc/postfix/dh_param_1024.pem
  smtpd_tls_protocols = !SSLv2, !SSLv3
  smtpd_tls_mandatory_protocols = !SSLv2, !SSLv3
  tls_preempt_cipherlist = yes
  tls_random_source = dev:/dev/urandom		
    %% NOTE: might want to have /dev/random here + Haveged
\end{verbatim}
  
For those users, who want to use ECC key exchange, it is possible to specify this via:
\begin{verbatim}
  smtpd_tls_eecdh_grade = ultra
\end{verbatim}

You can check the settings by specifying  smtpd\_tls\_loglevel = 1 and then check the selected ciphers with the following command:
\begin{verbatim}
$ zegrep "TLS connection established from.*with cipher" /var/log/mail.log | \
> awk '{printf("%s %s %s %s\n", $12, $13, $14, $15)}' | sort | uniq -c | sort -n
      1 SSLv3 with cipher DHE-RSA-AES256-SHA
     23 TLSv1.2 with cipher DHE-RSA-AES256-GCM-SHA384
     60 TLSv1 with cipher ECDHE-RSA-AES256-SHA
    270 TLSv1.2 with cipher ECDHE-RSA-AES256-GCM-SHA384
    335 TLSv1 with cipher DHE-RSA-AES256-SHA
\end{verbatim}

Source: \url{http://www.postfix.org/TLS_README.html}

\subsubsection{SMTP: opportunistic TLS}
% do we need to documment starttls in detail?
%\subsubsection{starttls?}

\subsection{SSH}

\begin{verbatim}
	RSAAuthentication yes
	PermitRootLogin no
	StrictModes yes
	Ciphers aes256-ctr
	MACs hmac-sha2-512,hmac-sha2-256,hmac-ripemd160
\end{verbatim}

Note: older linux systems won't support SHA2, PuTTY does not support RIPE-MD160.

\subsection{OpenVPN}

\subsection{IPSec}

\subsection{PGP}

\subsection{PRNG settings}
\label{section:PRNG}

