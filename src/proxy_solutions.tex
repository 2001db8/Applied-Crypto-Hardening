\subsection{Intercepting proxy solutions and reverse proxies}

Within enterprise networks and corporations with increased levels of paranoia or at least some defined security requirements it is common, NOT to allow direct connections to the public internet.

For this reason proxy-solutions are installed, to intercept and (hopefully also) scan the traffic for potential threats within the sessions.

As soon as one wants to establish an encrypted connection to a server, there are three choices:

\begin{itemize}
\item Block the connection, because it cannot be scanned for threats
\item Bypass the threat-mitigation and pass the encrypted session to the client, which results in a situation where malicious content is transferred directly to the client without visibility for the security system.
\item Intercept (i.e. terminate) the session at the proxy, scan there and re-encrypt the session towards the client.
\end{itemize}

While the last solution might be the most "up to date", it arises a new front in the context of this paper, because the most secure part of a client's connection could only be within the corporate network, if the proxy-server handles the connection to the destination server in an insecure manner.

Conclusio: Don't forget to check your proxy solutions ssl-capabilities. Also do so for your reverse-proxies!

\subsubsection{squid}
\todo{Write}
%% http://forum.pfsense.org/index.php?topic=63262.0

\begin{lstlisting}[breaklines]
		NO_SSLv2    Disallow the use of SSLv2
		NO_SSLv3    Disallow the use of SSLv3
		NO_TLSv1    Disallow the use of TLSv1.0
		NO_TLSv1_1  Disallow the use of TLSv1.1
		NO_TLSv1_2  Disallow the use of TLSv1.2
		SINGLE_DH_USE
				Always create a new key when using temporary/ephemeral
				DH key exchanges
\end{lstlisting}

\subsubsection{Bluecoat}
\todo{sure?}

