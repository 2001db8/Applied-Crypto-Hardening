\section{A note on Elliptic Curve Cryptography}
\label{section:EllipticCurveCryptography}

%\epigraph{``Mathematics is the queen of the sciences and number theory is the queen of mathematics.''}{-- Carl Friedrich Gauss}

\epigraph{``Everyone knows what a curve is, until he has studied enough
mathematics to become confused through the countless number of possible
exceptions.''}{-- Felix Klein }

\todo{cm says: factoring primes is impossible by definition!}
Elliptic Curve Cryptography (simply called ECC from now on) is a branch of
cryptography that emerged in the mid-1980s.  The security of the RSA
algorithm is based on the assumption that factoring large numbers is infeasible.
Likewise, the security of ECC, DH and DSA is based on the discrete logarithm
problem\cite{Wikipedia:Discrete,McC90,WR13}.  Finding the discrete logarithm of
an elliptic curve from its public base point is thought to be infeasible. This
is known as the Elliptic Curve Discrete Logarithm Problem (ECDLP). ECC and the
underlying mathematical foundation are not easy to understand - luckily, there
have been some great introductions on the topic lately
\footnote{\url{http://arstechnica.com/security/2013/10/a-relatively-easy-to-understand-primer-on-elliptic-curve-cryptography}}
\footnote{\url{https://www.imperialviolet.org/2010/12/04/ecc.html}}
\footnote{\url{http://www.isg.rhul.ac.uk/~sdg/ecc.html}}.

ECC provides for much stronger security with less computationally expensive
operations in comparison to traditional asymmetric algorithms (See the Section
\ref{section:keylengths}).


The security of ECC relies on the elliptic curves and curve points chosen as
parameters for the algorithm in question. Well before the NSA-leak scandal
there has been a lot of discussion regarding these parameters and their
potential subversion. A part of the discussion involved recommended sets of
curves and curve points chosen by different standardization bodies such as the
National Institute of Standards and Technology (NIST)
\footnote{\url{http://www.nist.gov}} which were later widely implemented in
most common crypto libraries. Those parameters came under question repeatedly
from cryptographers\cite{BL13,Sch13b,W13}.  At the time of writing, there is
ongoing research as to the security of various ECC parameters\cite{DJBSC}.
Most software configured to rely on ECC (be it client or server) is not able to
promote or black-list certain curves. It is the hope of the authors that such
functionality will be deployed widely soon.  The authors of this paper include
configurations and recommendations with and without ECC - the reader may choose
to adopt those settings as he finds best suited to his environment. The authors
will not make this decision for the reader.


\textbf{A word of warning:} One should get familiar with ECC, different curves
and parameters if one chooses to adopt ECC configurations. Since there is much
discussion on the security of ECC, flawed settings might very well compromise
the security of the entire system!

%% mention different attacks on ECC besides flawed parameters!

