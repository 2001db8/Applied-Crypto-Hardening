\section{Overview}
\label{sec:TheoryOverview}


\epigraph{``The balance between freedom and security is a delicate one.''}{-- Mark Udall, americal politician}

This chapter provides the necessary background information on why chapter \ref{chapter:PracticalSettings} recommended \textit{cipher string B}.

\vskip 0.5em
We start off by explaining the structure of cipher strings in section \ref (Architecture) and define some terms. Next we present \textit{Cipher String A} and \textit{Cipher String B} in sections \ref XXX and \ref YYY.

\vskip 0.5em
After that, the following sections deal with Random number generators, keylengths, current issues in ECC, a note of warning on SHA-1 and some comments on Diffie Hellman key exchanges. All of this is important in understanding why certain choices were made for \textit{Cipher String A and B}. For most system administrators, the question of compatibility is one of the most pressing ones. Having the freedom to be compatible with any client (even running on outdated operating systems) of course, reduces the security of our cipher strings. We address these topics in section \ref{XXX}. 
All these sections will allow a system administrator to balance his or her needs for strong encryption with useability and compatibility.

\vskip 0.5em

Last but not least, we finish this chapter by talking about issues in PKIs, Certificate Authorities and on hardening a PKI. Note that these last few topics deserve a book on their own. Hence this guide can only mention a few current topics in this area.

