\section{A note on SHA-1}
\label{section:SHA}


In the last years several weaknesses have been shown for SHA-1. In
particular, collisions on SHA-1 can be found using $2^{63}$ operations, and
recent results even indicate a lower complexity. Therefore,
ECRYPT II and NIST recommend against using SHA-1 for generating digital
signatures and for other applications that require collision resistance.
The use of SHA-1 in message authentication, e.g. HMAC, is not
immediately threatened.

We recommend using SHA-2 whenever available. Since SHA-2 is not
supported by older versions of TLS, SHA-1 can be used for message
authentication if a higher compatibility with a more diverse set of
clients is needed.


Our configurations A and B reflect this. While configuration A does not include
SHA-1, configuration B does and thus is more compatible with a wider range of
clients.
