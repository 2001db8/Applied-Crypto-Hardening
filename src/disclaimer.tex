\newpage
\section{Disclaimer and scope}
\label{section:disclaimer}

\epigraph{``A chain is no stronger than its weakest link, and life is after all a chain''}{William James}
\epigraph{``Encryption works. Properly implemented strong crypto systems are
one of the few things that you can rely on. Unfortunately, endpoint security is
so terrifically weak that NSA can frequently find ways around it.''}{Edward
Snowden, answering questions live on the Guardian's
website\cite{snowdenGuardianGreenwald}}


This guide specifically does not address physical security, protecting software
and hardware against exploits, basic IT security housekeeping, information
assurance techniques, traffic analysis attacks, issues with key-roll over and
key management, securing client PCs and mobile devices (theft, loss), proper
OPSec\footnote{\url{http://en.wikipedia.org/wiki/Operations_security}}, social
engineering attacks, anti-tempest\cite{Wikipedia:Tempest} attack techniques,
protecting against different side-channel attacks (timing--, cache timing--,
differential fault analysis, differential power analysis or power monitoring
attacks), downgrade attacks, jamming the encrypted channel or other similar
attacks which are typically employed to circumvent strong encryption.  The
authors can not overstate the importance of these other techniques.  Interested
readers are advised to read about these attacks in detail since they give a lot
of insight into other parts of cryptography engineering which need to be dealt
with.\footnote{An easy to read yet very insightful recent example is the
"FLUSH+RELOAD" technique \cite{yarom2013flush+} for leaking cryptographic keys
from one virtual machine to another via L3 cache timing attacks.}

This guide does not talk much about the well-known insecurities of trusting a
public-key infrastructure (PKI)\footnote{Interested readers are referred to
\url{https://bugzilla.mozilla.org/show_bug.cgi?id=647959} or
\url{http://www.heise.de/security/meldung/Der-ehrliche-Achmed-bittet-um-Vertrauen-1231083.html}
(german) which brings the problem of trusting PKIs right to the point}. Nor
does this text fully explain how to run your own Certificate Authority (CA). 


Most of this zoo of information security issues are addressed in the very
comprehensive book ``Security Engineering'' by Ross Anderson
\cite{anderson2008security}. 

For some experts in cryptography this text might seem too informal. However, we
strive to keep the language as non-technical as possible and fitting for our
target audience: system administrators who can collectively improve the
security level for all of their users. 



\epigraph{``Security is a process, not a product.''}{Bruce Schneier}

This guide can only describe what the authors currently
\emph{believe} to be the best settings based on their personal experience and
after intensive cross checking with literature and experts. For a complete list
of people who reviewed this paper, see the \nameref{section:Reviewers}.
Even though multiple specialists reviewed the guide, the authors can give
\emph{no guarantee whatsoever} that they made the right recommendations. Keep in
mind that tomorrow there might be new attacks on some ciphers and many of the
recommendations in this guide might turn out to be wrong. Security is a
process.


We therefore recommend that system administrators keep up to date with recent
topics in IT security and cryptography. 


In this sense, this guide is very focused on getting the cipher strings done
right even though there is much more to do in order to make a system more
secure.  We the authors, need this document as much as the reader needs it.

\paragraph{Scope:}
\label{section:Scope}

In this guide, we restricted ourselves to:
\begin{itemize}
\item Internet-facing services
\item Commonly used services
\item Devices which are used in business environments (this specifically excludes XBoxes, Playstations and similar consumer devices)
\item OpenSSL 
\end{itemize}

We explicitly excluded:
\begin{itemize}
\item Specialized systems (such as medical devices, most embedded systems, etc.)
\item Wireless Access Points
\item Smart-cards/chip cards
%\item Advice on running a PKI or a CA
%\item Services which should be run only in an internal network and never face the Internet.
\end{itemize}

