\section{Disclaimer}
\label{section:disclaimer}
This guide specifically does not address physical security issues, protecting software and hardware against exploits, basic IT security housekeeping tasks, anti-tempest\footnote{\url{https://en.wikipedia.org/wiki/Tempest\_(codename)}} attack techniques, protecting against side-channel attacks or other similar attacks which are usually employed to circumvent strong encryption. The authors can not overstate the importance of these other techniques. 

This guide can only describe what the authors currently \emph{believe} to be the best settings based on their personal experience. This guide was cross checked by multiple people. For a complete list, see the section ``reviewers''. Even though, multiple specialists reviewed the guide, the authors can give \emph{no guarantee} whatsover that they made the right recommendations. Keep in mind that tomorrow there might be new attacks on some ciphers and many of the recommendations in this guide might turn out to be wrong.


%% should we keep that sentence?
%% The authors do not know XXX FIXME XXX list things we don't know which affect the guide? XXX

Nevertheless, ignoring the problem at hand and keeping outdated settings for SSL, SSH, PGP is not an option. We the authors, need this document as much as the gentle reader needs it.

\todo{Aaron: mention downgrade attacks, jamming until they give up and go plaintext is the best cryptanalsysis technique}
